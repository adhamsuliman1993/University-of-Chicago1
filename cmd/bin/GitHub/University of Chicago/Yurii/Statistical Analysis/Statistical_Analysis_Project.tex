\documentclass[]{article}
\usepackage{lmodern}
\usepackage{amssymb,amsmath}
\usepackage{ifxetex,ifluatex}
\usepackage{fixltx2e} % provides \textsubscript
\ifnum 0\ifxetex 1\fi\ifluatex 1\fi=0 % if pdftex
  \usepackage[T1]{fontenc}
  \usepackage[utf8]{inputenc}
\else % if luatex or xelatex
  \ifxetex
    \usepackage{mathspec}
  \else
    \usepackage{fontspec}
  \fi
  \defaultfontfeatures{Ligatures=TeX,Scale=MatchLowercase}
\fi
% use upquote if available, for straight quotes in verbatim environments
\IfFileExists{upquote.sty}{\usepackage{upquote}}{}
% use microtype if available
\IfFileExists{microtype.sty}{%
\usepackage{microtype}
\UseMicrotypeSet[protrusion]{basicmath} % disable protrusion for tt fonts
}{}
\usepackage[margin=1in]{geometry}
\usepackage{hyperref}
\hypersetup{unicode=true,
            pdftitle={Final Project MSCA 31007 Statistical Analysis},
            pdfborder={0 0 0},
            breaklinks=true}
\urlstyle{same}  % don't use monospace font for urls
\usepackage{color}
\usepackage{fancyvrb}
\newcommand{\VerbBar}{|}
\newcommand{\VERB}{\Verb[commandchars=\\\{\}]}
\DefineVerbatimEnvironment{Highlighting}{Verbatim}{commandchars=\\\{\}}
% Add ',fontsize=\small' for more characters per line
\usepackage{framed}
\definecolor{shadecolor}{RGB}{248,248,248}
\newenvironment{Shaded}{\begin{snugshade}}{\end{snugshade}}
\newcommand{\KeywordTok}[1]{\textcolor[rgb]{0.13,0.29,0.53}{\textbf{#1}}}
\newcommand{\DataTypeTok}[1]{\textcolor[rgb]{0.13,0.29,0.53}{#1}}
\newcommand{\DecValTok}[1]{\textcolor[rgb]{0.00,0.00,0.81}{#1}}
\newcommand{\BaseNTok}[1]{\textcolor[rgb]{0.00,0.00,0.81}{#1}}
\newcommand{\FloatTok}[1]{\textcolor[rgb]{0.00,0.00,0.81}{#1}}
\newcommand{\ConstantTok}[1]{\textcolor[rgb]{0.00,0.00,0.00}{#1}}
\newcommand{\CharTok}[1]{\textcolor[rgb]{0.31,0.60,0.02}{#1}}
\newcommand{\SpecialCharTok}[1]{\textcolor[rgb]{0.00,0.00,0.00}{#1}}
\newcommand{\StringTok}[1]{\textcolor[rgb]{0.31,0.60,0.02}{#1}}
\newcommand{\VerbatimStringTok}[1]{\textcolor[rgb]{0.31,0.60,0.02}{#1}}
\newcommand{\SpecialStringTok}[1]{\textcolor[rgb]{0.31,0.60,0.02}{#1}}
\newcommand{\ImportTok}[1]{#1}
\newcommand{\CommentTok}[1]{\textcolor[rgb]{0.56,0.35,0.01}{\textit{#1}}}
\newcommand{\DocumentationTok}[1]{\textcolor[rgb]{0.56,0.35,0.01}{\textbf{\textit{#1}}}}
\newcommand{\AnnotationTok}[1]{\textcolor[rgb]{0.56,0.35,0.01}{\textbf{\textit{#1}}}}
\newcommand{\CommentVarTok}[1]{\textcolor[rgb]{0.56,0.35,0.01}{\textbf{\textit{#1}}}}
\newcommand{\OtherTok}[1]{\textcolor[rgb]{0.56,0.35,0.01}{#1}}
\newcommand{\FunctionTok}[1]{\textcolor[rgb]{0.00,0.00,0.00}{#1}}
\newcommand{\VariableTok}[1]{\textcolor[rgb]{0.00,0.00,0.00}{#1}}
\newcommand{\ControlFlowTok}[1]{\textcolor[rgb]{0.13,0.29,0.53}{\textbf{#1}}}
\newcommand{\OperatorTok}[1]{\textcolor[rgb]{0.81,0.36,0.00}{\textbf{#1}}}
\newcommand{\BuiltInTok}[1]{#1}
\newcommand{\ExtensionTok}[1]{#1}
\newcommand{\PreprocessorTok}[1]{\textcolor[rgb]{0.56,0.35,0.01}{\textit{#1}}}
\newcommand{\AttributeTok}[1]{\textcolor[rgb]{0.77,0.63,0.00}{#1}}
\newcommand{\RegionMarkerTok}[1]{#1}
\newcommand{\InformationTok}[1]{\textcolor[rgb]{0.56,0.35,0.01}{\textbf{\textit{#1}}}}
\newcommand{\WarningTok}[1]{\textcolor[rgb]{0.56,0.35,0.01}{\textbf{\textit{#1}}}}
\newcommand{\AlertTok}[1]{\textcolor[rgb]{0.94,0.16,0.16}{#1}}
\newcommand{\ErrorTok}[1]{\textcolor[rgb]{0.64,0.00,0.00}{\textbf{#1}}}
\newcommand{\NormalTok}[1]{#1}
\usepackage{graphicx,grffile}
\makeatletter
\def\maxwidth{\ifdim\Gin@nat@width>\linewidth\linewidth\else\Gin@nat@width\fi}
\def\maxheight{\ifdim\Gin@nat@height>\textheight\textheight\else\Gin@nat@height\fi}
\makeatother
% Scale images if necessary, so that they will not overflow the page
% margins by default, and it is still possible to overwrite the defaults
% using explicit options in \includegraphics[width, height, ...]{}
\setkeys{Gin}{width=\maxwidth,height=\maxheight,keepaspectratio}
\IfFileExists{parskip.sty}{%
\usepackage{parskip}
}{% else
\setlength{\parindent}{0pt}
\setlength{\parskip}{6pt plus 2pt minus 1pt}
}
\setlength{\emergencystretch}{3em}  % prevent overfull lines
\providecommand{\tightlist}{%
  \setlength{\itemsep}{0pt}\setlength{\parskip}{0pt}}
\setcounter{secnumdepth}{5}
% Redefines (sub)paragraphs to behave more like sections
\ifx\paragraph\undefined\else
\let\oldparagraph\paragraph
\renewcommand{\paragraph}[1]{\oldparagraph{#1}\mbox{}}
\fi
\ifx\subparagraph\undefined\else
\let\oldsubparagraph\subparagraph
\renewcommand{\subparagraph}[1]{\oldsubparagraph{#1}\mbox{}}
\fi

%%% Use protect on footnotes to avoid problems with footnotes in titles
\let\rmarkdownfootnote\footnote%
\def\footnote{\protect\rmarkdownfootnote}

%%% Change title format to be more compact
\usepackage{titling}

% Create subtitle command for use in maketitle
\newcommand{\subtitle}[1]{
  \posttitle{
    \begin{center}\large#1\end{center}
    }
}

\setlength{\droptitle}{-2em}
  \title{Final Project MSCA 31007 Statistical Analysis}
  \pretitle{\vspace{\droptitle}\centering\huge}
  \posttitle{\par}
  \author{}
  \preauthor{}\postauthor{}
  \predate{\centering\large\emph}
  \postdate{\par}
  \date{June 7, 2018}


\begin{document}
\maketitle

{
\setcounter{tocdepth}{3}
\tableofcontents
}
Step 1:

The first visualization looks at all of the data with relation to the
predictors(bonds). All of the bonds start at a yield of around 15\%, and
they follow the same general trend.There are periods of positive and
negative slopes, but the general trend seems to be going towards a
percent yield of below 5\%. The second visualization has all of the
predictors found in the first visualization, but it also includes Output
1. Output 1 seems to follow the same general trend of the predictors,
but it begins to deviate from the preditors at around the index of 3000.
The output 1 even falls below 0\% which would be an interesting concept
for a bond.

Step 2: Output1\textasciitilde{}USGG3M

\begin{Shaded}
\begin{Highlighting}[]
\NormalTok{Input1.Linear.Model<-}\KeywordTok{lm}\NormalTok{(Output1}\OperatorTok{~}\NormalTok{USGG3M,AssignmentData)}
\KeywordTok{c}\NormalTok{(}\DataTypeTok{Total.Variance=}\KeywordTok{var}\NormalTok{(AssignmentData[,}\DecValTok{8}\NormalTok{]),}\DataTypeTok{Unexplained.Variance=}\KeywordTok{summary}\NormalTok{(Input1.Linear.Model)}\OperatorTok{$}\NormalTok{sigma}\OperatorTok{^}\DecValTok{2}\NormalTok{)}
\end{Highlighting}
\end{Shaded}

\begin{verbatim}
##       Total.Variance Unexplained.Variance 
##            76.804438             2.857058
\end{verbatim}

\begin{Shaded}
\begin{Highlighting}[]
\KeywordTok{summary}\NormalTok{(Input1.Linear.Model)}\OperatorTok{$}\NormalTok{coefficients}
\end{Highlighting}
\end{Shaded}

\begin{verbatim}
##               Estimate  Std. Error   t value Pr(>|t|)
## (Intercept) -11.723184 0.031369514 -373.7126        0
## USGG3M        2.507561 0.005410479  463.4638        0
\end{verbatim}

\begin{Shaded}
\begin{Highlighting}[]
\KeywordTok{c}\NormalTok{(}\DataTypeTok{R2=}\NormalTok{(}\KeywordTok{summary}\NormalTok{(Input1.Linear.Model)}\OperatorTok{$}\NormalTok{r.squared),}\DataTypeTok{R2.adj=}\NormalTok{(}\KeywordTok{summary}\NormalTok{(Input1.Linear.Model)}\OperatorTok{$}\NormalTok{adj.r.squared))}
\end{Highlighting}
\end{Shaded}

\begin{verbatim}
##        R2    R2.adj 
## 0.9628054 0.9628009
\end{verbatim}

\begin{Shaded}
\begin{Highlighting}[]
\KeywordTok{matplot}\NormalTok{(AssignmentData[,}\DecValTok{8}\NormalTok{],}\DataTypeTok{type=}\StringTok{"l"}\NormalTok{,}\DataTypeTok{xaxt=}\StringTok{"n"}\NormalTok{)}
  \KeywordTok{lines}\NormalTok{(Input1.Linear.Model}\OperatorTok{$}\NormalTok{fitted.values,}\DataTypeTok{col=}\StringTok{"red"}\NormalTok{)}
\end{Highlighting}
\end{Shaded}

\includegraphics{Statistical_Analysis_Project_files/figure-latex/unnamed-chunk-2-1.pdf}
The Linear Regresseion model here takes the Output against the USGG3M.
The unexplained variance amounts to 2.8 out of 76.8 which shows that
these two data sets seem to be fairly similar. All of the models below
will have a total variance of 76.8 which is stemming from Output 1. The
unexplained variance will be reduced to the extent that the predictor is
capable of explaining the variance. From the summary of the
coefficients, the Pr(\textgreater{}t) = 0 shows that there is
significant evidence that the coefficient and intercept are not equal to
0. As expected, R2 and R2.adj are equal because there is only one
predictor in the model. R2.adj is used to penalize correlation when more
than one predictor is being utilized. R2 is not reduced as more
predictors are utilized. R2 and R2.adj will be expected to be the same
for all models utilizing one predictor.

Output1\textasciitilde{}USGG6M

\begin{Shaded}
\begin{Highlighting}[]
\NormalTok{Input2.Linear.Model<-}\KeywordTok{lm}\NormalTok{(Output1}\OperatorTok{~}\NormalTok{USGG6M,AssignmentData)}
\KeywordTok{c}\NormalTok{(}\DataTypeTok{Total.Variance=}\KeywordTok{var}\NormalTok{(AssignmentData[,}\DecValTok{8}\NormalTok{]),}\DataTypeTok{Unexplained.Variance=}\KeywordTok{summary}\NormalTok{(Input2.Linear.Model)}\OperatorTok{$}\NormalTok{sigma}\OperatorTok{^}\DecValTok{2}\NormalTok{)}
\end{Highlighting}
\end{Shaded}

\begin{verbatim}
##       Total.Variance Unexplained.Variance 
##            76.804438             1.967321
\end{verbatim}

\begin{Shaded}
\begin{Highlighting}[]
\KeywordTok{summary}\NormalTok{(Input2.Linear.Model)}\OperatorTok{$}\NormalTok{coefficients}
\end{Highlighting}
\end{Shaded}

\begin{verbatim}
##               Estimate Std. Error   t value Pr(>|t|)
## (Intercept) -12.097528 0.02646897 -457.0457        0
## USGG6M        2.497235 0.00444452  561.8683        0
\end{verbatim}

\begin{Shaded}
\begin{Highlighting}[]
\KeywordTok{c}\NormalTok{(}\DataTypeTok{R2=}\NormalTok{(}\KeywordTok{summary}\NormalTok{(Input2.Linear.Model)}\OperatorTok{$}\NormalTok{r.squared),}\DataTypeTok{R2.adj=}\NormalTok{(}\KeywordTok{summary}\NormalTok{(Input2.Linear.Model)}\OperatorTok{$}\NormalTok{adj.r.squared))}
\end{Highlighting}
\end{Shaded}

\begin{verbatim}
##        R2    R2.adj 
## 0.9743884 0.9743853
\end{verbatim}

\begin{Shaded}
\begin{Highlighting}[]
\KeywordTok{matplot}\NormalTok{(AssignmentData[,}\DecValTok{8}\NormalTok{],}\DataTypeTok{type=}\StringTok{"l"}\NormalTok{,}\DataTypeTok{xaxt=}\StringTok{"n"}\NormalTok{)}
\KeywordTok{lines}\NormalTok{(Input2.Linear.Model}\OperatorTok{$}\NormalTok{fitted.values,}\DataTypeTok{col=}\StringTok{"red"}\NormalTok{)}
\end{Highlighting}
\end{Shaded}

\includegraphics{Statistical_Analysis_Project_files/figure-latex/unnamed-chunk-3-1.pdf}
The Linear Regresseion model here takes the Output against the USGG6M.
The unexplained variance amounts to 1.9 out of 76.8 which shows that
these two data sets seem to be fairly similar. From the summary of the
coefficients, the Pr(\textgreater{}t) = 0 shows that there is
significant evidence that the coefficient and intercept are not equal to
0.

Output1\textasciitilde{}USGG2YR

\begin{Shaded}
\begin{Highlighting}[]
\NormalTok{Input3.Linear.Model<-}\KeywordTok{lm}\NormalTok{(Output1}\OperatorTok{~}\NormalTok{USGG2YR,AssignmentData)}
\KeywordTok{c}\NormalTok{(}\DataTypeTok{Total.Variance=}\KeywordTok{var}\NormalTok{(AssignmentData[,}\DecValTok{8}\NormalTok{]),}\DataTypeTok{Unexplained.Variance=}\KeywordTok{summary}\NormalTok{(Input3.Linear.Model)}\OperatorTok{$}\NormalTok{sigma}\OperatorTok{^}\DecValTok{2}\NormalTok{)}
\end{Highlighting}
\end{Shaded}

\begin{verbatim}
##       Total.Variance Unexplained.Variance 
##           76.8044379            0.2588092
\end{verbatim}

\begin{Shaded}
\begin{Highlighting}[]
\KeywordTok{summary}\NormalTok{(Input3.Linear.Model)}\OperatorTok{$}\NormalTok{coefficients}
\end{Highlighting}
\end{Shaded}

\begin{verbatim}
##               Estimate  Std. Error   t value Pr(>|t|)
## (Intercept) -13.055775 0.010031274 -1301.507        0
## USGG2YR       2.400449 0.001532178  1566.691        0
\end{verbatim}

\begin{Shaded}
\begin{Highlighting}[]
\KeywordTok{c}\NormalTok{(}\DataTypeTok{R2=}\NormalTok{(}\KeywordTok{summary}\NormalTok{(Input3.Linear.Model)}\OperatorTok{$}\NormalTok{r.squared),}\DataTypeTok{R2.adj=}\NormalTok{(}\KeywordTok{summary}\NormalTok{(Input3.Linear.Model)}\OperatorTok{$}\NormalTok{adj.r.squared))}
\end{Highlighting}
\end{Shaded}

\begin{verbatim}
##        R2    R2.adj 
## 0.9966307 0.9966303
\end{verbatim}

\begin{Shaded}
\begin{Highlighting}[]
\KeywordTok{matplot}\NormalTok{(AssignmentData[,}\DecValTok{8}\NormalTok{],}\DataTypeTok{type=}\StringTok{"l"}\NormalTok{,}\DataTypeTok{xaxt=}\StringTok{"n"}\NormalTok{)}
\KeywordTok{lines}\NormalTok{(Input3.Linear.Model}\OperatorTok{$}\NormalTok{fitted.values,}\DataTypeTok{col=}\StringTok{"red"}\NormalTok{)}
\end{Highlighting}
\end{Shaded}

\includegraphics{Statistical_Analysis_Project_files/figure-latex/unnamed-chunk-4-1.pdf}
The Linear Regresseion model here takes the Output against the USGG2YR.
The unexplained variance amounts to .25 out of 76.8 which shows that
these two data sets seem to be fairly similar. From the summary of the
coefficients, the Pr(\textgreater{}t) = 0 shows that there is
significant evidence that the coefficient and intercept are not equal to
0.

Output1\textasciitilde{}USGG3YR

\begin{Shaded}
\begin{Highlighting}[]
\NormalTok{Input4.Linear.Model<-}\KeywordTok{lm}\NormalTok{(Output1}\OperatorTok{~}\NormalTok{USGG3YR,AssignmentData)}
\KeywordTok{c}\NormalTok{(}\DataTypeTok{Total.Variance=}\KeywordTok{var}\NormalTok{(AssignmentData[,}\DecValTok{8}\NormalTok{]),}\DataTypeTok{Unexplained.Variance=}\KeywordTok{summary}\NormalTok{(Input4.Linear.Model)}\OperatorTok{$}\NormalTok{sigma}\OperatorTok{^}\DecValTok{2}\NormalTok{)}
\end{Highlighting}
\end{Shaded}

\begin{verbatim}
##       Total.Variance Unexplained.Variance 
##            76.804438             0.159657
\end{verbatim}

\begin{Shaded}
\begin{Highlighting}[]
\KeywordTok{summary}\NormalTok{(Input4.Linear.Model)}\OperatorTok{$}\NormalTok{coefficients}
\end{Highlighting}
\end{Shaded}

\begin{verbatim}
##               Estimate  Std. Error   t value Pr(>|t|)
## (Intercept) -13.861618 0.008213688 -1687.624        0
## USGG3YR       2.455793 0.001230358  1995.999        0
\end{verbatim}

\begin{Shaded}
\begin{Highlighting}[]
\KeywordTok{c}\NormalTok{(}\DataTypeTok{R2=}\NormalTok{(}\KeywordTok{summary}\NormalTok{(Input4.Linear.Model)}\OperatorTok{$}\NormalTok{r.squared),}\DataTypeTok{R2.adj=}\NormalTok{(}\KeywordTok{summary}\NormalTok{(Input4.Linear.Model)}\OperatorTok{$}\NormalTok{adj.r.squared))}
\end{Highlighting}
\end{Shaded}

\begin{verbatim}
##        R2    R2.adj 
## 0.9979215 0.9979213
\end{verbatim}

\begin{Shaded}
\begin{Highlighting}[]
\KeywordTok{matplot}\NormalTok{(AssignmentData[,}\DecValTok{8}\NormalTok{],}\DataTypeTok{type=}\StringTok{"l"}\NormalTok{,}\DataTypeTok{xaxt=}\StringTok{"n"}\NormalTok{)}
\KeywordTok{lines}\NormalTok{(Input4.Linear.Model}\OperatorTok{$}\NormalTok{fitted.values,}\DataTypeTok{col=}\StringTok{"red"}\NormalTok{)}
\end{Highlighting}
\end{Shaded}

\includegraphics{Statistical_Analysis_Project_files/figure-latex/unnamed-chunk-5-1.pdf}
The Linear Regresseion model here takes the Output against the USGG3YR.
The unexplained variance amounts to .16 out of 76.8 which shows that
these two data sets seem to be fairly similar. From the summary of the
coefficients, the Pr(\textgreater{}t) = 0 shows that there is
significant evidence that the coefficient and intercept are not equal to
0.

Output1\textasciitilde{}USGG5YR

\begin{Shaded}
\begin{Highlighting}[]
\NormalTok{Input5.Linear.Model<-}\KeywordTok{lm}\NormalTok{(Output1}\OperatorTok{~}\NormalTok{USGG5YR,AssignmentData)}
\KeywordTok{head}\NormalTok{(AssignmentData,}\DecValTok{3}\NormalTok{)}
\end{Highlighting}
\end{Shaded}

\begin{verbatim}
##          USGG3M USGG6M USGG2YR USGG3YR USGG5YR USGG10YR USGG30YR  Output1 Easing Tightening
## 1/5/1981  13.52  13.09  12.289   12.28  12.294   12.152   11.672 18.01553     NA         NA
## 1/6/1981  13.58  13.16  12.429   12.31  12.214   12.112   11.672 18.09140     NA         NA
## 1/7/1981  14.50  13.90  12.929   12.78  12.614   12.382   11.892 19.44731     NA         NA
\end{verbatim}

\begin{Shaded}
\begin{Highlighting}[]
\KeywordTok{c}\NormalTok{(}\DataTypeTok{Total.Variance=}\KeywordTok{var}\NormalTok{(AssignmentData[,}\DecValTok{8}\NormalTok{]),}\DataTypeTok{Unexplained.Variance=}\KeywordTok{summary}\NormalTok{(Input5.Linear.Model)}\OperatorTok{$}\NormalTok{sigma}\OperatorTok{^}\DecValTok{2}\NormalTok{)}
\end{Highlighting}
\end{Shaded}

\begin{verbatim}
##       Total.Variance Unexplained.Variance 
##           76.8044379            0.6382572
\end{verbatim}

\begin{Shaded}
\begin{Highlighting}[]
\KeywordTok{summary}\NormalTok{(Input5.Linear.Model)}\OperatorTok{$}\NormalTok{coefficients}
\end{Highlighting}
\end{Shaded}

\begin{verbatim}
##               Estimate  Std. Error   t value Pr(>|t|)
## (Intercept) -15.436649 0.017818763 -866.3143        0
## USGG5YR       2.568742 0.002581214  995.1678        0
\end{verbatim}

\begin{Shaded}
\begin{Highlighting}[]
\KeywordTok{c}\NormalTok{(}\DataTypeTok{R2=}\NormalTok{(}\KeywordTok{summary}\NormalTok{(Input5.Linear.Model)}\OperatorTok{$}\NormalTok{r.squared),}\DataTypeTok{R2.adj=}\NormalTok{(}\KeywordTok{summary}\NormalTok{(Input5.Linear.Model)}\OperatorTok{$}\NormalTok{adj.r.squared))}
\end{Highlighting}
\end{Shaded}

\begin{verbatim}
##        R2    R2.adj 
## 0.9916908 0.9916898
\end{verbatim}

\begin{Shaded}
\begin{Highlighting}[]
\KeywordTok{matplot}\NormalTok{(AssignmentData[,}\DecValTok{8}\NormalTok{],}\DataTypeTok{type=}\StringTok{"l"}\NormalTok{,}\DataTypeTok{xaxt=}\StringTok{"n"}\NormalTok{)}
\KeywordTok{lines}\NormalTok{(Input5.Linear.Model}\OperatorTok{$}\NormalTok{fitted.values,}\DataTypeTok{col=}\StringTok{"red"}\NormalTok{)}
\end{Highlighting}
\end{Shaded}

\includegraphics{Statistical_Analysis_Project_files/figure-latex/unnamed-chunk-6-1.pdf}
The Linear Regresseion model here takes the Output against the USGG3YR.
The unexplained variance amounts to .63 out of 76.8 which shows that
these two data sets seem to be fairly similar. From the summary of the
coefficients, the Pr(\textgreater{}t) = 0 shows that there is
significant evidence that the coefficient and intercept are not equal to
0.

Output1\textasciitilde{}USGG10YR

\begin{Shaded}
\begin{Highlighting}[]
\NormalTok{Input6.Linear.Model<-}\KeywordTok{lm}\NormalTok{(Output1}\OperatorTok{~}\NormalTok{USGG10YR,AssignmentData)}
\KeywordTok{c}\NormalTok{(}\DataTypeTok{Total.Variance=}\KeywordTok{var}\NormalTok{(AssignmentData[,}\DecValTok{8}\NormalTok{]),}\DataTypeTok{Unexplained.Variance=}\KeywordTok{summary}\NormalTok{(Input6.Linear.Model)}\OperatorTok{$}\NormalTok{sigma}\OperatorTok{^}\DecValTok{2}\NormalTok{)}
\end{Highlighting}
\end{Shaded}

\begin{verbatim}
##       Total.Variance Unexplained.Variance 
##            76.804438             2.366752
\end{verbatim}

\begin{Shaded}
\begin{Highlighting}[]
\KeywordTok{summary}\NormalTok{(Input6.Linear.Model)}\OperatorTok{$}\NormalTok{coefficients}
\end{Highlighting}
\end{Shaded}

\begin{verbatim}
##               Estimate  Std. Error   t value Pr(>|t|)
## (Intercept) -18.063370 0.039181738 -461.0150        0
## USGG10YR      2.786991 0.005455089  510.8975        0
\end{verbatim}

\begin{Shaded}
\begin{Highlighting}[]
\KeywordTok{c}\NormalTok{(}\DataTypeTok{R2=}\NormalTok{(}\KeywordTok{summary}\NormalTok{(Input6.Linear.Model)}\OperatorTok{$}\NormalTok{r.squared),}\DataTypeTok{R2.adj=}\NormalTok{(}\KeywordTok{summary}\NormalTok{(Input6.Linear.Model)}\OperatorTok{$}\NormalTok{adj.r.squared))}
\end{Highlighting}
\end{Shaded}

\begin{verbatim}
##        R2    R2.adj 
## 0.9691884 0.9691847
\end{verbatim}

\begin{Shaded}
\begin{Highlighting}[]
\KeywordTok{matplot}\NormalTok{(AssignmentData[,}\DecValTok{8}\NormalTok{],}\DataTypeTok{type=}\StringTok{"l"}\NormalTok{,}\DataTypeTok{xaxt=}\StringTok{"n"}\NormalTok{)}
\KeywordTok{lines}\NormalTok{(Input6.Linear.Model}\OperatorTok{$}\NormalTok{fitted.values,}\DataTypeTok{col=}\StringTok{"red"}\NormalTok{)}
\end{Highlighting}
\end{Shaded}

\includegraphics{Statistical_Analysis_Project_files/figure-latex/unnamed-chunk-7-1.pdf}
The Linear Regresseion model here takes the Output against the USGG3YR.
The unexplained variance amounts to 2.4 out of 76.8 which shows that
these two data sets seem to be fairly similar. From the summary of the
coefficients, the Pr(\textgreater{}t) = 0 shows that there is
significant evidence that the coefficient and intercept are not equal to
0.

Output1\textasciitilde{}USGG30YR

\begin{Shaded}
\begin{Highlighting}[]
\NormalTok{Input7.Linear.Model<-}\KeywordTok{lm}\NormalTok{(Output1}\OperatorTok{~}\NormalTok{USGG30YR,AssignmentData)}
\KeywordTok{c}\NormalTok{(}\DataTypeTok{Total.Variance=}\KeywordTok{var}\NormalTok{(AssignmentData[,}\DecValTok{8}\NormalTok{]),}\DataTypeTok{Unexplained.Variance=}\KeywordTok{summary}\NormalTok{(Input7.Linear.Model)}\OperatorTok{$}\NormalTok{sigma}\OperatorTok{^}\DecValTok{2}\NormalTok{)}
\end{Highlighting}
\end{Shaded}

\begin{verbatim}
##       Total.Variance Unexplained.Variance 
##            76.804438             4.967286
\end{verbatim}

\begin{Shaded}
\begin{Highlighting}[]
\KeywordTok{summary}\NormalTok{(Input7.Linear.Model)}\OperatorTok{$}\NormalTok{coefficients}
\end{Highlighting}
\end{Shaded}

\begin{verbatim}
##               Estimate  Std. Error   t value Pr(>|t|)
## (Intercept) -21.085905 0.065596726 -321.4475        0
## USGG30YR      3.069561 0.008860263  346.4413        0
\end{verbatim}

\begin{Shaded}
\begin{Highlighting}[]
\KeywordTok{c}\NormalTok{(}\DataTypeTok{R2=}\NormalTok{(}\KeywordTok{summary}\NormalTok{(Input7.Linear.Model)}\OperatorTok{$}\NormalTok{r.squared),}\DataTypeTok{R2.adj=}\NormalTok{(}\KeywordTok{summary}\NormalTok{(Input7.Linear.Model)}\OperatorTok{$}\NormalTok{adj.r.squared))}
\end{Highlighting}
\end{Shaded}

\begin{verbatim}
##        R2    R2.adj 
## 0.9353333 0.9353255
\end{verbatim}

\begin{Shaded}
\begin{Highlighting}[]
\KeywordTok{matplot}\NormalTok{(AssignmentData[,}\DecValTok{8}\NormalTok{],}\DataTypeTok{type=}\StringTok{"l"}\NormalTok{,}\DataTypeTok{xaxt=}\StringTok{"n"}\NormalTok{)}
\KeywordTok{lines}\NormalTok{(Input7.Linear.Model}\OperatorTok{$}\NormalTok{fitted.values,}\DataTypeTok{col=}\StringTok{"red"}\NormalTok{)}
\end{Highlighting}
\end{Shaded}

\includegraphics{Statistical_Analysis_Project_files/figure-latex/unnamed-chunk-8-1.pdf}
The Linear Regresseion model here takes the Output against the USGG3YR.
The unexplained variance amounts to 4.9 out of 76.8 which shows that
these two data sets seem to be fairly similar. From the summary of our
coefficients, the Pr(\textgreater{}t) = 0 shows that there is
significant evidence that the coefficient and intercept are not equal to
0.

If one predictor must be selected from the above to best represent
Output 1, USGG3YR would be used because of it's minimal unexplained
variance and it's maximized R2. All of the models fitted values seemed
to correspond to the output 1 values quite well, which is why all the
models had such a high adj.R2 and a low amount of unexplained variance.

Collect all slopes and intercepts in one table and print this table. Try
to do it in one line using apply() function.

\begin{Shaded}
\begin{Highlighting}[]
\KeywordTok{apply}\NormalTok{(AssignmentData,}\DecValTok{2}\NormalTok{, }\ControlFlowTok{function}\NormalTok{(z) }\KeywordTok{lm}\NormalTok{(Output1}\OperatorTok{~}\NormalTok{z,AssignmentData)}\OperatorTok{$}\NormalTok{coefficients)}
\end{Highlighting}
\end{Shaded}

\begin{verbatim}
##                 USGG3M     USGG6M    USGG2YR    USGG3YR    USGG5YR   USGG10YR   USGG30YR
## (Intercept) -11.723184 -12.097528 -13.055775 -13.861618 -15.436649 -18.063370 -21.085905
## z             2.507561   2.497235   2.400449   2.455793   2.568742   2.786991   3.069561
##                   Output1   Easing Tightening
## (Intercept) -4.766731e-26 7.473583   9.986771
## z            1.000000e+00       NA         NA
\end{verbatim}

It's interesting to see that all of the bonds have negative intecepts
with positive slopes ranging from 2.4 to slightly above 3.

Step 3: Fit linear regression models using single output as input and
each ofthe original inputs as outputs. Collect all slopes and intercepts
in one table and print this table

\begin{Shaded}
\begin{Highlighting}[]
\KeywordTok{apply}\NormalTok{(AssignmentData,}\DecValTok{2}\NormalTok{, }\ControlFlowTok{function}\NormalTok{(z) }\KeywordTok{lm}\NormalTok{(z}\OperatorTok{~}\NormalTok{Output1,AssignmentData)}\OperatorTok{$}\NormalTok{coefficients)}
\end{Highlighting}
\end{Shaded}

\begin{verbatim}
##                USGG3M   USGG6M   USGG2YR   USGG3YR  USGG5YR  USGG10YR  USGG30YR       Output1
## (Intercept) 4.6751341 4.844370 5.4388879 5.6444580 6.009421 6.4813160 6.8693554 -4.766731e-26
## Output1     0.3839609 0.390187 0.4151851 0.4063541 0.386061 0.3477544 0.3047124  1.000000e+00
##                   Easing    Tightening
## (Intercept) 1.000000e+00  1.000000e+00
## Output1     2.310393e-16 -2.087244e-17
\end{verbatim}

When Output 1 is made to be the input, all of the intercepts for the
bonds are now positive ranging from 4.6 to 4.9 with positive slopes
ranging from .3 to .4.

Step 4: Estimate logistic regression using all inputs and the data on
FED tightening and easing cycles.

\begin{Shaded}
\begin{Highlighting}[]
\NormalTok{AssignmentDataLogistic<-}\KeywordTok{data.matrix}\NormalTok{(AssignmentData,}\DataTypeTok{rownames.force=}\StringTok{"automatic"}\NormalTok{)}
\NormalTok{EasingPeriods<-AssignmentDataLogistic[,}\DecValTok{9}\NormalTok{]}
\NormalTok{EasingPeriods[AssignmentDataLogistic[,}\DecValTok{9}\NormalTok{]}\OperatorTok{==}\DecValTok{1}\NormalTok{]<-}\DecValTok{0}
\NormalTok{TighteningPeriods<-AssignmentDataLogistic[,}\DecValTok{10}\NormalTok{]}
\KeywordTok{cbind}\NormalTok{(EasingPeriods,TighteningPeriods)[}\KeywordTok{c}\NormalTok{(}\DecValTok{550}\OperatorTok{:}\DecValTok{560}\NormalTok{,}\DecValTok{900}\OperatorTok{:}\DecValTok{910}\NormalTok{,}\DecValTok{970}\OperatorTok{:}\DecValTok{980}\NormalTok{),]}
\end{Highlighting}
\end{Shaded}

\begin{verbatim}
##            EasingPeriods TighteningPeriods
## 3/29/1983             NA                NA
## 3/30/1983             NA                NA
## 3/31/1983             NA                NA
## 4/4/1983              NA                 1
## 4/5/1983              NA                 1
## 4/6/1983              NA                 1
## 4/7/1983              NA                 1
## 4/8/1983              NA                 1
## 4/11/1983             NA                 1
## 4/12/1983             NA                 1
## 4/13/1983             NA                 1
## 8/27/1984             NA                 1
## 8/28/1984             NA                 1
## 8/29/1984             NA                 1
## 8/30/1984             NA                 1
## 8/31/1984             NA                 1
## 9/4/1984              NA                NA
## 9/5/1984              NA                NA
## 9/6/1984              NA                NA
## 9/7/1984              NA                NA
## 9/10/1984             NA                NA
## 9/11/1984             NA                NA
## 12/10/1984             0                NA
## 12/11/1984             0                NA
## 12/12/1984             0                NA
## 12/13/1984             0                NA
## 12/14/1984             0                NA
## 12/17/1984             0                NA
## 12/18/1984             0                NA
## 12/19/1984             0                NA
## 12/20/1984             0                NA
## 12/21/1984             0                NA
## 12/24/1984             0                NA
\end{verbatim}

\begin{Shaded}
\begin{Highlighting}[]
\NormalTok{All.NAs<-}\KeywordTok{is.na}\NormalTok{(EasingPeriods)}\OperatorTok{&}\KeywordTok{is.na}\NormalTok{(TighteningPeriods)}
\NormalTok{AssignmentDataLogistic.EasingTighteningOnly<-AssignmentDataLogistic}
\NormalTok{AssignmentDataLogistic.EasingTighteningOnly[,}\DecValTok{9}\NormalTok{]<-EasingPeriods}
\NormalTok{AssignmentDataLogistic.EasingTighteningOnly<-AssignmentDataLogistic.EasingTighteningOnly[}\OperatorTok{!}\NormalTok{All.NAs,]}
\NormalTok{AssignmentDataLogistic.EasingTighteningOnly[}\KeywordTok{is.na}\NormalTok{(AssignmentDataLogistic.EasingTighteningOnly[,}\DecValTok{10}\NormalTok{]),}\DecValTok{10}\NormalTok{]<-}\DecValTok{0}
\KeywordTok{matplot}\NormalTok{(AssignmentDataLogistic.EasingTighteningOnly[,}\OperatorTok{-}\KeywordTok{c}\NormalTok{(}\DecValTok{9}\NormalTok{,}\DecValTok{10}\NormalTok{)],}\DataTypeTok{type=}\StringTok{"l"}\NormalTok{,}\DataTypeTok{ylab=}\StringTok{"Data and Binary Fed Mode"}\NormalTok{)}
\KeywordTok{lines}\NormalTok{(AssignmentDataLogistic.EasingTighteningOnly[,}\DecValTok{10}\NormalTok{]}\OperatorTok{*}\DecValTok{20}\NormalTok{,}\DataTypeTok{col=}\StringTok{"red"}\NormalTok{)}
\end{Highlighting}
\end{Shaded}

\includegraphics{Statistical_Analysis_Project_files/figure-latex/unnamed-chunk-10-1.pdf}
It is quite visible that when the treasurey has decided to go through a
period of tightening within the past, interest rates would rise
gradually. This makes sense in that if there is not hat much free
cashflow in the market place, investors should receive higher returns on
their free capital because cash is in demand. The initial tightening
period takes place between index 250 and about index 600. This is the
longest tightening period within the dataset, and there also seems to be
substantial growth with the rates and especially output 1. The same
relation seems to occur during the tightening period between 1100 and
1200 and the tightening period between 1300 and 1550.

\begin{Shaded}
\begin{Highlighting}[]
\NormalTok{LogisticModel.TighteningEasing_3M<-}\KeywordTok{glm}\NormalTok{(AssignmentDataLogistic.EasingTighteningOnly[,}\DecValTok{10}\NormalTok{]}\OperatorTok{~}
\StringTok{                                      }\NormalTok{AssignmentDataLogistic.EasingTighteningOnly[,}\DecValTok{1}\NormalTok{],}\DataTypeTok{family=}\KeywordTok{binomial}\NormalTok{(}\DataTypeTok{link=}\NormalTok{logit))}
\KeywordTok{summary}\NormalTok{(LogisticModel.TighteningEasing_3M)}
\end{Highlighting}
\end{Shaded}

\begin{verbatim}
## 
## Call:
## glm(formula = AssignmentDataLogistic.EasingTighteningOnly[, 10] ~ 
##     AssignmentDataLogistic.EasingTighteningOnly[, 1], family = binomial(link = logit))
## 
## Deviance Residuals: 
##     Min       1Q   Median       3Q      Max  
## -1.4239  -0.9014  -0.7737   1.3548   1.6743  
## 
## Coefficients:
##                                                  Estimate Std. Error z value Pr(>|z|)    
## (Intercept)                                      -2.15256    0.17328 -12.422   <2e-16 ***
## AssignmentDataLogistic.EasingTighteningOnly[, 1]  0.18638    0.02144   8.694   <2e-16 ***
## ---
## Signif. codes:  0 '***' 0.001 '**' 0.01 '*' 0.05 '.' 0.1 ' ' 1
## 
## (Dispersion parameter for binomial family taken to be 1)
## 
##     Null deviance: 2983.5  on 2357  degrees of freedom
## Residual deviance: 2904.8  on 2356  degrees of freedom
## AIC: 2908.8
## 
## Number of Fisher Scoring iterations: 4
\end{verbatim}

\begin{Shaded}
\begin{Highlighting}[]
\KeywordTok{matplot}\NormalTok{(AssignmentDataLogistic.EasingTighteningOnly[,}\OperatorTok{-}\KeywordTok{c}\NormalTok{(}\DecValTok{9}\NormalTok{,}\DecValTok{10}\NormalTok{)],}\DataTypeTok{type=}\StringTok{"l"}\NormalTok{,}\DataTypeTok{ylab=}\StringTok{"Data and Fitted Values"}\NormalTok{)}
\KeywordTok{lines}\NormalTok{(AssignmentDataLogistic.EasingTighteningOnly[,}\DecValTok{10}\NormalTok{]}\OperatorTok{*}\DecValTok{20}\NormalTok{,}\DataTypeTok{col=}\StringTok{"red"}\NormalTok{)}
\KeywordTok{lines}\NormalTok{(LogisticModel.TighteningEasing_3M}\OperatorTok{$}\NormalTok{fitted.values}\OperatorTok{*}\DecValTok{20}\NormalTok{,}\DataTypeTok{col=}\StringTok{"green"}\NormalTok{)}
\end{Highlighting}
\end{Shaded}

\includegraphics{Statistical_Analysis_Project_files/figure-latex/unnamed-chunk-11-1.pdf}
The fitted values of the glm is the green line above. The fitted values
for the Easining and Tightening model against the USGG3M follows the
trend of the rest of the bonds in that it has a positive slope in
periods of tightening while having a negative slope in periods of
easing. The only period where this doesn't apply seems to be towards the
very begining of our data set and right before the first tigteining
period where the fitted values have a positive slope for a short time
span. This green line is the probability created through the logistic
regression which will determine if the date was in either an easining or
a tightening period. Periods of tightening are identified by the red
bars that flatten out at 20 on the y-axis. By taking the fitted line and
multiplying it by 20, it can be thought of as a probability that it is
within a tighetining period.

\begin{Shaded}
\begin{Highlighting}[]
\NormalTok{AssignmentDataLogistic.EasingTighteningOnly<-}\KeywordTok{data.frame}\NormalTok{(AssignmentDataLogistic.EasingTighteningOnly)}
\NormalTok{LogisticModel.TighteningEasing_All<-}\KeywordTok{glm}\NormalTok{(AssignmentDataLogistic.EasingTighteningOnly[,}\DecValTok{10}\NormalTok{]}\OperatorTok{~}\NormalTok{.,AssignmentDataLogistic.EasingTighteningOnly[,}\OperatorTok{-}\KeywordTok{c}\NormalTok{(}\DecValTok{8}\NormalTok{,}\DecValTok{9}\NormalTok{,}\DecValTok{10}\NormalTok{)],}\DataTypeTok{family=}\KeywordTok{binomial}\NormalTok{(}\DataTypeTok{link=}\NormalTok{logit))}
\KeywordTok{summary}\NormalTok{(LogisticModel.TighteningEasing_All)}\OperatorTok{$}\NormalTok{aic}
\end{Highlighting}
\end{Shaded}

\begin{verbatim}
## [1] 2645.579
\end{verbatim}

\begin{Shaded}
\begin{Highlighting}[]
\KeywordTok{summary}\NormalTok{(LogisticModel.TighteningEasing_All)}\OperatorTok{$}\NormalTok{coefficients[,}\KeywordTok{c}\NormalTok{(}\DecValTok{1}\NormalTok{,}\DecValTok{4}\NormalTok{)]}
\end{Highlighting}
\end{Shaded}

\begin{verbatim}
##               Estimate     Pr(>|z|)
## (Intercept) -4.7551928 2.784283e-28
## USGG3M      -3.3456116 4.073045e-36
## USGG6M       4.1558535 1.422964e-28
## USGG2YR      3.9460296 1.751687e-07
## USGG3YR     -3.4642455 2.080617e-04
## USGG5YR     -3.2115319 3.786229e-05
## USGG10YR    -0.9705444 3.202140e-01
## USGG30YR     3.3253517 6.036041e-08
\end{verbatim}

\begin{Shaded}
\begin{Highlighting}[]
\KeywordTok{matplot}\NormalTok{(AssignmentDataLogistic.EasingTighteningOnly[,}\OperatorTok{-}\KeywordTok{c}\NormalTok{(}\DecValTok{9}\NormalTok{,}\DecValTok{10}\NormalTok{)],}\DataTypeTok{type=}\StringTok{"l"}\NormalTok{,}\DataTypeTok{ylab=}\StringTok{"Results of Logistic Regression"}\NormalTok{)}
\KeywordTok{lines}\NormalTok{(AssignmentDataLogistic.EasingTighteningOnly[,}\DecValTok{10}\NormalTok{]}\OperatorTok{*}\DecValTok{20}\NormalTok{,}\DataTypeTok{col=}\StringTok{"red"}\NormalTok{)}
\KeywordTok{lines}\NormalTok{(LogisticModel.TighteningEasing_All}\OperatorTok{$}\NormalTok{fitted.values}\OperatorTok{*}\DecValTok{20}\NormalTok{,}\DataTypeTok{col=}\StringTok{"green"}\NormalTok{)}
\end{Highlighting}
\end{Shaded}

\includegraphics{Statistical_Analysis_Project_files/figure-latex/unnamed-chunk-12-1.pdf}
USGG3M has the smallest Pr(\textgreater{}\textbar{}z\textbar{}) which
means it has the largest affect upon the glm. Due to such a large affect
upon the glm model that takes into account easing and tightening, it may
be wise to utilize USGG3m in the predictive model. USGG10YR has the
largest Pr(\textgreater{}\textbar{}z\textbar{}) which means it has the
smallest effect upon the glm. In the previous model, a logistic
regression of Output 1 was taken against USGG3M. In this model, a
logistic regression of Output 1 is taken against all bonds. What's
interesting is that even though USGG3M has the largest affect of any of
the bonds, when the other bonds are included, there is a much larger
variance within the fitted values. The fitted values still follow the
trend spoken to in the previous model.

Lop.odds

\begin{Shaded}
\begin{Highlighting}[]
\NormalTok{Log.Odds<-}\KeywordTok{predict}\NormalTok{(LogisticModel.TighteningEasing_All)}
\KeywordTok{plot}\NormalTok{(Log.Odds,}\DataTypeTok{type=}\StringTok{"l"}\NormalTok{)}
\end{Highlighting}
\end{Shaded}

\includegraphics{Statistical_Analysis_Project_files/figure-latex/unnamed-chunk-13-1.pdf}

\begin{Shaded}
\begin{Highlighting}[]
\NormalTok{Probabilities<-}\DecValTok{1}\OperatorTok{/}\NormalTok{(}\KeywordTok{exp}\NormalTok{(}\OperatorTok{-}\NormalTok{Log.Odds)}\OperatorTok{+}\DecValTok{1}\NormalTok{)}
\KeywordTok{plot}\NormalTok{(LogisticModel.TighteningEasing_All}\OperatorTok{$}\NormalTok{fitted.values,}\DataTypeTok{type=}\StringTok{"l"}\NormalTok{,}\DataTypeTok{ylab=}\StringTok{"Fitted Values & Log-Odds"}\NormalTok{)}
\KeywordTok{lines}\NormalTok{(Probabilities,}\DataTypeTok{col=}\StringTok{"red"}\NormalTok{)}
\end{Highlighting}
\end{Shaded}

\includegraphics{Statistical_Analysis_Project_files/figure-latex/unnamed-chunk-13-2.pdf}
The log odds uses the fitted values of the Tightening periods against
all bonds and puts it into a predictive model. By looking at the
probabilities, the Y axis is reduced to a scale between 0 adn 1 and
keeps a similar figure.

Step 5: Compare linear regression models with different combinations of
predicotrs. Selec tthe best combination.\\
Estiamte the full model by using all 7 predictors

\begin{Shaded}
\begin{Highlighting}[]
\NormalTok{AssignmentDataRegressionComparison<-}\KeywordTok{data.matrix}\NormalTok{(AssignmentData[,}\OperatorTok{-}\KeywordTok{c}\NormalTok{(}\DecValTok{9}\NormalTok{,}\DecValTok{10}\NormalTok{)],}\DataTypeTok{rownames.force=}\StringTok{"automatic"}\NormalTok{)}
\NormalTok{AssignmentDataRegressionComparison<-AssignmentData[,}\OperatorTok{-}\KeywordTok{c}\NormalTok{(}\DecValTok{9}\NormalTok{,}\DecValTok{10}\NormalTok{)]}
\NormalTok{RegressionModelComparison.Full<-}\KeywordTok{lm}\NormalTok{(AssignmentDataRegressionComparison[,}\DecValTok{8}\NormalTok{]}\OperatorTok{~}\NormalTok{.,AssignmentDataRegressionComparison[,}\OperatorTok{-}\DecValTok{8}\NormalTok{])}
\KeywordTok{summary}\NormalTok{(RegressionModelComparison.Full)}\OperatorTok{$}\NormalTok{coefficients}
\end{Highlighting}
\end{Shaded}

\begin{verbatim}
##                Estimate   Std. Error       t value Pr(>|t|)
## (Intercept) -14.9041591 1.056850e-10 -141024294891        0
## USGG3M        0.3839609 9.860401e-11    3893968285        0
## USGG6M        0.3901870 1.500111e-10    2601053702        0
## USGG2YR       0.4151851 2.569451e-10    1615851177        0
## USGG3YR       0.4063541 3.299038e-10    1231735395        0
## USGG5YR       0.3860610 2.618339e-10    1474449865        0
## USGG10YR      0.3477544 2.800269e-10    1241860763        0
## USGG30YR      0.3047124 1.566487e-10    1945195584        0
\end{verbatim}

\begin{Shaded}
\begin{Highlighting}[]
\KeywordTok{c}\NormalTok{(}\DataTypeTok{R2=}\KeywordTok{summary}\NormalTok{(RegressionModelComparison.Full)}\OperatorTok{$}\NormalTok{r.squared, }\DataTypeTok{Adjusted.R2=}\KeywordTok{summary}\NormalTok{(RegressionModelComparison.Full)}\OperatorTok{$}\NormalTok{adj.r.squared)}
\end{Highlighting}
\end{Shaded}

\begin{verbatim}
##          R2 Adjusted.R2 
##           1           1
\end{verbatim}

\begin{Shaded}
\begin{Highlighting}[]
\KeywordTok{summary}\NormalTok{(RegressionModelComparison.Full)}\OperatorTok{$}\NormalTok{df}
\end{Highlighting}
\end{Shaded}

\begin{verbatim}
## [1]    8 8292    8
\end{verbatim}

Q: Interpret the fitted Model. How good is the fit? How significant are
the paramters? A: Because R2 and asjursted.R2 are 1, the model is a
perfect fit which would mean the model is overfitting. This is not an
ideal scenario in statistics. Since all of the
Pr(\textgreater{}\textbar{}t\textbar{}) =0, the current model decrees
that all parameters are significant. All predictors shouldn't be used
because as a perfect model in Analytics isn't very useful for future
predictive analysis.

Estimate the null model by including only the intercept

\begin{Shaded}
\begin{Highlighting}[]
\NormalTok{RegressionModelComparison.Null<-}\KeywordTok{lm}\NormalTok{(AssignmentDataRegressionComparison[,}\DecValTok{8}\NormalTok{]}\OperatorTok{~+}\DecValTok{1}\NormalTok{,AssignmentDataRegressionComparison[,}\OperatorTok{-}\DecValTok{8}\NormalTok{])}
\KeywordTok{summary}\NormalTok{(RegressionModelComparison.Null)}
\end{Highlighting}
\end{Shaded}

\begin{verbatim}
## 
## Call:
## lm(formula = AssignmentDataRegressionComparison[, 8] ~ +1, data = AssignmentDataRegressionComparison[, 
##     -8])
## 
## Residuals:
##     Min      1Q  Median      3Q     Max 
## -13.173  -6.509  -0.415   4.860  27.298 
## 
## Coefficients:
##             Estimate Std. Error t value Pr(>|t|)
## (Intercept) 1.42e-11   9.62e-02       0        1
## 
## Residual standard error: 8.764 on 8299 degrees of freedom
\end{verbatim}

\begin{Shaded}
\begin{Highlighting}[]
\KeywordTok{c}\NormalTok{(}\DataTypeTok{R2=}\KeywordTok{summary}\NormalTok{(RegressionModelComparison.Null)}\OperatorTok{$}\NormalTok{r.squared, }\DataTypeTok{Adjusted.R2=}\KeywordTok{summary}\NormalTok{(RegressionModelComparison.Null)}\OperatorTok{$}\NormalTok{adj.r.squared, }\DataTypeTok{df=}\KeywordTok{summary}\NormalTok{(RegressionModelComparison.Null)}\OperatorTok{$}\NormalTok{df)}
\end{Highlighting}
\end{Shaded}

\begin{verbatim}
##          R2 Adjusted.R2         df1         df2         df3 
##           0           0           1        8299           1
\end{verbatim}

\begin{Shaded}
\begin{Highlighting}[]
\KeywordTok{anova}\NormalTok{(RegressionModelComparison.Full,RegressionModelComparison.Null)}
\end{Highlighting}
\end{Shaded}

\begin{verbatim}
## Analysis of Variance Table
## 
## Model 1: AssignmentDataRegressionComparison[, 8] ~ USGG3M + USGG6M + USGG2YR + 
##     USGG3YR + USGG5YR + USGG10YR + USGG30YR
## Model 2: AssignmentDataRegressionComparison[, 8] ~ +1
##   Res.Df    RSS Df Sum of Sq        F    Pr(>F)    
## 1   8292      0                                    
## 2   8299 637400 -7   -637400 3.73e+22 < 2.2e-16 ***
## ---
## Signif. codes:  0 '***' 0.001 '**' 0.01 '*' 0.05 '.' 0.1 ' ' 1
\end{verbatim}

Why summary(RegressionModelComparison.Null) does not show R2? When a
regression model is taken for an Output against the intercept, the model
doesn't take into account any of the predictors. Without the predictors,
the model cannot be explained simply by the intercept therefore there is
no correlation to show for. This is why R2 and adjusted.R2 are equal to
zero. When the anova is taken between the two models,
RegressionModelComparison.Full has a large RSS because it explains the
model fully, but it also has 7 degrees of freedom less because it ulizes
all predictors.

The best combination is attempted to be found by utilizing the add1()
process

\begin{Shaded}
\begin{Highlighting}[]
\NormalTok{(myScope<-}\KeywordTok{names}\NormalTok{(AssignmentDataRegressionComparison)[}\OperatorTok{-}\KeywordTok{which}\NormalTok{(}\KeywordTok{names}\NormalTok{(AssignmentDataRegressionComparison)}\OperatorTok{==}\StringTok{"Output1"}\NormalTok{)])}
\end{Highlighting}
\end{Shaded}

\begin{verbatim}
## [1] "USGG3M"   "USGG6M"   "USGG2YR"  "USGG3YR"  "USGG5YR"  "USGG10YR" "USGG30YR"
\end{verbatim}

\begin{Shaded}
\begin{Highlighting}[]
\KeywordTok{anova}\NormalTok{(RegressionModelComparison.Null)}
\end{Highlighting}
\end{Shaded}

\begin{verbatim}
## Analysis of Variance Table
## 
## Response: AssignmentDataRegressionComparison[, 8]
##             Df Sum Sq Mean Sq F value Pr(>F)
## Residuals 8299 637400  76.804
\end{verbatim}

\begin{Shaded}
\begin{Highlighting}[]
\KeywordTok{add1}\NormalTok{(RegressionModelComparison.Null,}\DataTypeTok{scope=}\NormalTok{myScope)}
\end{Highlighting}
\end{Shaded}

\begin{verbatim}
## Single term additions
## 
## Model:
## AssignmentDataRegressionComparison[, 8] ~ +1
##          Df Sum of Sq    RSS    AIC
## <none>                637400  36033
## USGG3M    1    613692  23708   8715
## USGG6M    1    621075  16325   5618
## USGG2YR   1    635252   2148 -11217
## USGG3YR   1    636075   1325 -15226
## USGG5YR   1    632104   5296  -3725
## USGG10YR  1    617761  19639   7153
## USGG30YR  1    596181  41219  13306
\end{verbatim}

\begin{Shaded}
\begin{Highlighting}[]
\NormalTok{(myScope<-myScope[}\OperatorTok{-}\KeywordTok{which}\NormalTok{(myScope}\OperatorTok{==}\StringTok{"USGG3YR"}\NormalTok{)])}
\end{Highlighting}
\end{Shaded}

\begin{verbatim}
## [1] "USGG3M"   "USGG6M"   "USGG2YR"  "USGG5YR"  "USGG10YR" "USGG30YR"
\end{verbatim}

\begin{Shaded}
\begin{Highlighting}[]
\NormalTok{RegressionModelComparison.3yr<-}\KeywordTok{lm}\NormalTok{(Output1}\OperatorTok{~}\DecValTok{1}\OperatorTok{+}\NormalTok{USGG3YR, AssignmentDataRegressionComparison)}
\KeywordTok{add1}\NormalTok{(RegressionModelComparison.3yr,myScope)}
\end{Highlighting}
\end{Shaded}

\begin{verbatim}
## Single term additions
## 
## Model:
## Output1 ~ 1 + USGG3YR
##          Df Sum of Sq     RSS    AIC
## <none>                1324.83 -15226
## USGG3M    1    407.98  916.85 -18279
## USGG6M    1    270.17 1054.66 -17117
## USGG2YR   1     82.93 1241.91 -15761
## USGG5YR   1     20.94 1303.89 -15356
## USGG10YR  1     64.05 1260.78 -15636
## USGG30YR  1    100.79 1224.04 -15881
\end{verbatim}

\begin{Shaded}
\begin{Highlighting}[]
\NormalTok{myScope<-myScope[}\OperatorTok{-}\KeywordTok{which}\NormalTok{(myScope}\OperatorTok{==}\StringTok{"USGG3M"}\NormalTok{)]}
\NormalTok{RegressionModelComparison.3yr.3m<-}\KeywordTok{lm}\NormalTok{(Output1}\OperatorTok{~}\DecValTok{1}\OperatorTok{+}\NormalTok{USGG3YR}\OperatorTok{+}\NormalTok{USGG3M, AssignmentDataRegressionComparison)}
\KeywordTok{add1}\NormalTok{(RegressionModelComparison.3yr.3m, myScope)}
\end{Highlighting}
\end{Shaded}

\begin{verbatim}
## Single term additions
## 
## Model:
## Output1 ~ 1 + USGG3YR + USGG3M
##          Df Sum of Sq    RSS    AIC
## <none>                916.85 -18279
## USGG6M    1    139.91 776.95 -19652
## USGG2YR   1    176.99 739.86 -20058
## USGG5YR   1    736.49 180.36 -31773
## USGG10YR  1    864.29  52.56 -42007
## USGG30YR  1    863.62  53.23 -41902
\end{verbatim}

\begin{Shaded}
\begin{Highlighting}[]
\NormalTok{RegressionModelComparison.3yr.3m.30yr<-}\KeywordTok{lm}\NormalTok{(Output1}\OperatorTok{~}\DecValTok{1}\OperatorTok{+}\NormalTok{USGG3YR}\OperatorTok{+}\NormalTok{USGG3M}\OperatorTok{+}\NormalTok{USGG30YR, AssignmentDataRegressionComparison)}
\NormalTok{myScope<-myScope[}\OperatorTok{-}\KeywordTok{which}\NormalTok{(myScope}\OperatorTok{==}\StringTok{"USGG30YR"}\NormalTok{)]}
\KeywordTok{add1}\NormalTok{(RegressionModelComparison.3yr.3m.30yr, myScope)}
\end{Highlighting}
\end{Shaded}

\begin{verbatim}
## Single term additions
## 
## Model:
## Output1 ~ 1 + USGG3YR + USGG3M + USGG30YR
##          Df Sum of Sq    RSS    AIC
## <none>                53.230 -41902
## USGG6M    1   26.5758 26.654 -47641
## USGG2YR   1    7.5370 45.693 -43167
## USGG5YR   1    9.5669 43.663 -43544
## USGG10YR  1   10.9537 42.276 -43812
\end{verbatim}

\begin{Shaded}
\begin{Highlighting}[]
\NormalTok{RegressionModelComparison.3yr.3m.30yr.6m<-}\KeywordTok{lm}\NormalTok{(Output1}\OperatorTok{~}\DecValTok{1}\OperatorTok{+}\NormalTok{USGG3YR}\OperatorTok{+}\NormalTok{USGG3M}\OperatorTok{+}\NormalTok{USGG30YR}\OperatorTok{+}\NormalTok{USGG6M, AssignmentDataRegressionComparison)}
\NormalTok{myScope<-myScope[}\OperatorTok{-}\KeywordTok{which}\NormalTok{(myScope}\OperatorTok{==}\StringTok{"USGG6M"}\NormalTok{)]}
\KeywordTok{add1}\NormalTok{(RegressionModelComparison.3yr.3m.30yr.6m, myScope)}
\end{Highlighting}
\end{Shaded}

\begin{verbatim}
## Single term additions
## 
## Model:
## Output1 ~ 1 + USGG3YR + USGG3M + USGG30YR + USGG6M
##          Df Sum of Sq     RSS    AIC
## <none>                26.6542 -47641
## USGG2YR   1    0.0556 26.5986 -47656
## USGG5YR   1   17.6312  9.0230 -56629
## USGG10YR  1   17.2110  9.4433 -56251
\end{verbatim}

The objective is to find a combination of predictors that produces the
lowest AIC. Using the add1() process, the first most important predictor
is USGG3YR. When USGG3YR is taken out of our predictors and the add1()
process is run again, the most important predictor is USGG3M. When
USGG3YR and USGG3m are taken out of he predictors and the add1() process
is run again, the most important predictor is USGG30YR.

Using the MuMIn Package, all combinations of predictors' AICs can be
seen to determine which combination of predictors will lead to the most
accurate model while attempting to reserve degrees of freedom.

\begin{Shaded}
\begin{Highlighting}[]
\NormalTok{lm.test<-}\KeywordTok{lm}\NormalTok{(Output1}\OperatorTok{~}\NormalTok{., AssignmentDataRegressionComparison, }\DataTypeTok{na.action=}\StringTok{'na.fail'}\NormalTok{)}
\KeywordTok{summary}\NormalTok{(}\KeywordTok{model.avg}\NormalTok{(}\KeywordTok{dredge}\NormalTok{(lm.test)))}
\end{Highlighting}
\end{Shaded}

\begin{verbatim}
## Fixed term is "(Intercept)"
\end{verbatim}

\begin{verbatim}
## 
## Call:
## model.avg(object = dredge(lm.test))
## 
## Component model call: 
## lm(formula = Output1 ~ <128 unique rhs>, data = AssignmentDataRegressionComparison, 
##      na.action = na.fail)
## 
## Component models: 
##         df    logLik       AICc    delta weight
## 1234567  9 156526.06 -313034.10      0.0      1
## 123467   8  20238.68  -40461.34 272572.8      0
## 234567   8  20170.73  -40325.44 272708.7      0
## 123457   8  18745.84  -37475.65 275558.5      0
## 134567   8  17985.75  -35955.48 277078.6      0
## 23467    7  17330.88  -34647.75 278386.4      0
## 34567    7  16543.38  -33072.75 279961.4      0
## 124567   8  16446.09  -32876.17 280157.9      0
## 13457    7  16354.45  -32694.89 280339.2      0
## 12457    7  15737.16  -31460.30 281573.8      0
## 12467    7  15637.22  -31260.43 281773.7      0
## 123456   8  14034.49  -28052.97 284981.1      0
## 12346    7  13670.88  -27327.74 285706.4      0
## 14567    7  13353.62  -26693.22 286340.9      0
## 1457     6  13340.73  -26669.45 286364.7      0
## 23456    7  13020.92  -26027.83 287006.3      0
## 12345    7  12876.94  -25739.87 287294.2      0
## 12347    7  12847.23  -25680.44 287353.7      0
## 1247     6  12827.07  -25642.13 287392.0      0
## 2346     6  12742.30  -25472.59 287561.5      0
## 12456    7  12691.31  -25368.61 287665.5      0
## 1246     6  12563.03  -25114.05 287920.0      0
## 1245     6  12298.23  -24584.44 288449.7      0
## 23457    7  12056.89  -24099.77 288934.3      0
## 3457     6  12048.22  -24084.43 288949.7      0
## 1234     6  11216.36  -22420.72 290613.4      0
## 124      5  11211.32  -22412.64 290621.5      0
## 123567   8  10685.34  -21354.67 291679.4      0
## 13567    7  10502.01  -20990.00 292044.1      0
## 23567    7  10394.31  -20774.61 292259.5      0
## 3567     6  10269.39  -20526.76 292507.3      0
## 13456    7  10233.89  -20453.76 292580.3      0
## 12367    7  10169.33  -20324.64 292709.5      0
## 12357    7  10155.92  -20297.83 292736.3      0
## 1345     6  10133.90  -20255.80 292778.3      0
## 1357     6  10097.33  -20182.65 292851.5      0
## 3456     6   9999.96  -19987.90 293046.2      0
## 2367     6   9913.34  -19814.67 293219.4      0
## 2345     6   9811.38  -19610.75 293423.4      0
## 12567    7   9272.05  -18530.08 294504.0      0
## 1456     6   9245.30  -18478.58 294555.5      0
## 145      5   9230.13  -18450.26 294583.8      0
## 1257     6   9205.88  -18399.74 294634.4      0
## 345      5   9177.77  -18345.53 294688.6      0
## 1267     6   9079.89  -18147.78 294886.3      0
## 1567     6   8960.01  -17908.02 295126.1      0
## 157      5   8955.09  -17900.16 295133.9      0
## 2357     6   8718.56  -17425.10 295609.0      0
## 357      5   8660.38  -17310.76 295723.3      0
## 1237     6   8583.79  -17155.57 295878.5      0
## 127      5   8486.64  -16963.28 296070.8      0
## 3467     6   8156.96  -16301.92 296732.2      0
## 13467    7   8157.25  -16300.49 296733.6      0
## 1467     6   7459.47  -14906.94 298127.2      0
## 367      5   7356.42  -14702.82 298331.3      0
## 1367     6   7356.59  -14701.16 298332.9      0
## 167      5   6643.89  -13277.78 299756.3      0
## 2347     6   6073.10  -12134.19 300899.9      0
## 234      5   5884.79  -11759.58 301274.5      0
## 24567    7   5129.54  -10245.06 302789.0      0
## 2456     6   5056.79  -10101.57 302932.5      0
## 237      5   4997.85   -9985.68 303048.4      0
## 2467     6   4787.23   -9562.45 303471.7      0
## 246      5   4690.45   -9370.89 303663.2      0
## 4567     6   4511.98   -9011.95 304022.2      0
## 467      5   4460.30   -8910.59 304123.5      0
## 456      5   4113.28   -8216.56 304817.5      0
## 1346     6   3912.61   -7813.21 305220.9      0
## 346      5   3857.91   -7705.81 305328.3      0
## 146      5   3737.79   -7465.57 305568.5      0
## 46       4   3432.00   -6856.00 306178.1      0
## 2567     6   3359.55   -6707.10 306327.0      0
## 1347     6   3330.45   -6648.89 306385.2      0
## 137      5   3319.59   -6629.18 306404.9      0
## 567      5   3148.48   -6286.95 306747.2      0
## 267      5   3074.48   -6138.95 306895.2      0
## 67       4   3071.92   -6135.83 306898.3      0
## 147      5   2606.61   -5203.21 307830.9      0
## 17       4   2553.65   -5099.29 307934.8      0
## 134      5   -456.32     922.66 313956.8      0
## 12356    7   -596.25    1206.52 314240.6      0
## 1235     6   -619.53    1251.07 314285.2      0
## 2356     6   -619.92    1251.85 314286.0      0
## 1256     6   -663.07    1338.15 314372.2      0
## 125      5   -667.10    1344.21 314378.3      0
## 235      5   -733.35    1476.71 314510.8      0
## 1236     6   -801.23    1614.47 314648.6      0
## 236      5   -824.49    1658.99 314693.1      0
## 23       4   -831.00    1670.01 314704.1      0
## 123      5   -830.87    1671.75 314705.9      0
## 126      5   -911.05    1832.10 314866.2      0
## 12       4  -1124.06    2256.13 315290.2      0
## 256      5  -1499.12    3008.25 316042.4      0
## 2457     6  -1639.19    3290.40 316324.5      0
## 245      5  -1744.38    3498.77 316532.9      0
## 347      5  -1869.60    3749.20 316783.3      0
## 457      5  -1947.34    3904.69 316938.8      0
## 26       4  -2211.92    4431.84 317465.9      0
## 37       4  -2389.78    4787.56 317821.7      0
## 14       4  -2549.22    5106.44 318140.5      0
## 257      5  -2594.10    5198.20 318232.3      0
## 45       4  -2634.48    5276.96 318311.1      0
## 57       4  -3215.59    6439.18 319473.3      0
## 1356     6  -3272.40    6556.80 319590.9      0
## 356      5  -3285.02    6580.05 319614.2      0
## 135      5  -3428.53    6867.06 319901.2      0
## 156      5  -3504.78    7019.57 320053.7      0
## 35       4  -3833.69    7675.39 320709.5      0
## 25       4  -3893.82    7795.64 320829.7      0
## 15       4  -3956.41    7920.83 320954.9      0
## 56       4  -4095.95    8199.90 321234.0      0
## 5        3  -4162.07    8330.14 321364.2      0
## 247      5  -4871.50    9753.00 322787.1      0
## 27       4  -5813.59   11635.19 324669.3      0
## 24       4  -6085.36   12178.73 325212.8      0
## 2        3  -6166.78   12339.57 325373.7      0
## 34       4  -6443.05   12894.10 325928.2      0
## 136      5  -7264.89   14539.79 327573.9      0
## 16       4  -7423.44   14854.89 327889.0      0
## 36       4  -7861.05   15730.11 328764.2      0
## 6        3  -9912.78   19831.57 332865.7      0
## 13       4 -11175.06   22358.12 335392.2      0
## 47       4 -13959.90   27927.80 340961.9      0
## 7        3 -14584.38   29174.76 342208.9      0
## 1        3 -15351.49   30708.99 343743.1      0
## 4        3 -16132.83   32271.66 345305.8      0
## 3        3 -18428.12   36862.23 349896.3      0
## (Null)   2 -29792.93   59589.86 372624.0      0
## 
## Term codes: 
## USGG10YR  USGG2YR USGG30YR   USGG3M  USGG3YR  USGG5YR   USGG6M 
##        1        2        3        4        5        6        7 
## 
## Model-averaged coefficients:  
## (full average) 
##               Estimate Std. Error Adjusted SE   z value Pr(>|z|)    
## (Intercept) -1.490e+01  1.057e-10   1.057e-10 1.410e+11   <2e-16 ***
## USGG10YR     3.478e-01  2.800e-10   2.801e-10 1.242e+09   <2e-16 ***
## USGG2YR      4.152e-01  2.569e-10   2.570e-10 1.616e+09   <2e-16 ***
## USGG30YR     3.047e-01  1.566e-10   1.567e-10 1.945e+09   <2e-16 ***
## USGG3M       3.840e-01  9.860e-11   9.862e-11 3.893e+09   <2e-16 ***
## USGG3YR      4.064e-01  3.299e-10   3.299e-10 1.232e+09   <2e-16 ***
## USGG5YR      3.861e-01  2.618e-10   2.619e-10 1.474e+09   <2e-16 ***
## USGG6M       3.902e-01  1.500e-10   1.500e-10 2.601e+09   <2e-16 ***
##  
## (conditional average) 
##               Estimate Std. Error Adjusted SE   z value Pr(>|z|)    
## (Intercept) -1.490e+01  1.057e-10   1.057e-10 1.410e+11   <2e-16 ***
## USGG10YR     3.478e-01  2.800e-10   2.801e-10 1.242e+09   <2e-16 ***
## USGG2YR      4.152e-01  2.569e-10   2.570e-10 1.616e+09   <2e-16 ***
## USGG30YR     3.047e-01  1.566e-10   1.567e-10 1.945e+09   <2e-16 ***
## USGG3M       3.840e-01  9.860e-11   9.862e-11 3.893e+09   <2e-16 ***
## USGG3YR      4.064e-01  3.299e-10   3.299e-10 1.232e+09   <2e-16 ***
## USGG5YR      3.861e-01  2.618e-10   2.619e-10 1.474e+09   <2e-16 ***
## USGG6M       3.902e-01  1.500e-10   1.500e-10 2.601e+09   <2e-16 ***
## ---
## Signif. codes:  0 '***' 0.001 '**' 0.01 '*' 0.05 '.' 0.1 ' ' 1
## 
## Relative variable importance: 
##                      USGG10YR USGG2YR USGG30YR USGG3M USGG3YR USGG5YR USGG6M
## Importance:           1        1       1        1      1       1       1    
## N containing models: 64       64      64       64     64      64      64
\end{verbatim}

\begin{Shaded}
\begin{Highlighting}[]
\NormalTok{my.model<-}\KeywordTok{lm}\NormalTok{(Output1}\OperatorTok{~}\DecValTok{1}\OperatorTok{+}\NormalTok{USGG10YR}\OperatorTok{+}\NormalTok{USGG2YR}\OperatorTok{+}\NormalTok{USGG3M, AssignmentDataRegressionComparison)}
\KeywordTok{cbind}\NormalTok{(}\StringTok{"Using MuMIn"}\NormalTok{=}\StringTok{ }\KeywordTok{AIC}\NormalTok{(my.model), }\StringTok{"Using Stepwise Model"}\NormalTok{ =}\StringTok{ }\KeywordTok{AIC}\NormalTok{(RegressionModelComparison.3yr.3m.30yr))}
\end{Highlighting}
\end{Shaded}

\begin{verbatim}
##      Using MuMIn Using Stepwise Model
## [1,]   -22412.65            -18345.53
\end{verbatim}

\begin{Shaded}
\begin{Highlighting}[]
\KeywordTok{cbind}\NormalTok{(}\StringTok{'Adjusted.R.2'}\NormalTok{=(}\KeywordTok{summary}\NormalTok{(my.model)}\OperatorTok{$}\NormalTok{adj.r.squared),}\StringTok{'Adjusted.R.2'}\NormalTok{=(}\KeywordTok{summary}\NormalTok{(RegressionModelComparison.3yr.3m.30yr)}\OperatorTok{$}\NormalTok{adj.r.squared ))}
\end{Highlighting}
\end{Shaded}

\begin{verbatim}
##      Adjusted.R.2 Adjusted.R.2
## [1,]    0.9999488    0.9999165
\end{verbatim}

If the add1() process is utilized, it would show that the USGG3YR is
first most important predictor. Using the MuMIn package, if only one
predictor had to be used, USGG3YR would be the best predictor as well.
The problem with the add1() process is that it doesn't take into account
combinations of predictors because it is a stepwise function. With the
MuMIn package, we are able to run all combinations of predictors to
determine which has the lowest AIC while reserving degrees of freedom.
From the table above, the AIC of the best combinations of predictors
with their respective degrees of freedome can bee seen below: DF AIC: 9
-313034 8 -40461 7 -34647 6 -26669 5 -22412 4 -6856

When determining the selection of the best predictors to utilize to
predict Output 1, R2 and R2.adjusted would not suffice when
distinguishing between models because they don't penalize larger data
sets enough to determine which model is truly the best. Utilizing AIC,
the lowest AIC was the model with df=9 where all predictors were used.
This should not be the chosen model becuase the model would be
overfitted. There is justification to utilize any other combination of
predictors because they all produce a high R2.adjusted while not
overfitting the model. A final decision to utilize 3 predictors was
decided upon because there is a massive reduction in AIC between the
model with 4 degrees of freedom and the model with 5 degrees of freedom.
Something that was very interesting is that the model with 5 degrees of
freedom didn't include USGG3YR as a predictor though it was considered
to be the most important predictor of all predictors utilizng the add1()
process if only one predictor were to be utilized.

BIC is an even stricter method that is used to differentiate predictors.
The equation for BIC is
-2log-likelihood+n\_parameters(log(n\_observations)). Below, the model
chosen utilizing MuMIn vs add1() are observed using BIC.

\begin{Shaded}
\begin{Highlighting}[]
\KeywordTok{c}\NormalTok{(}\StringTok{'Using MuMIn'}\NormalTok{=(}\KeywordTok{BIC}\NormalTok{(my.model)),}\StringTok{'Using add1()'}\NormalTok{=(}\KeywordTok{BIC}\NormalTok{(RegressionModelComparison.3yr.3m.30yr)))}
\end{Highlighting}
\end{Shaded}

\begin{verbatim}
##  Using MuMIn Using add1() 
##    -22377.53    -18310.41
\end{verbatim}

BIC still identifies the combination created utilizing the MuMIn package
as a better set of predictors than what would have been produced with
the add1() process.

\begin{Shaded}
\begin{Highlighting}[]
\KeywordTok{anova}\NormalTok{(my.model,RegressionModelComparison.3yr.3m.30yr)}
\end{Highlighting}
\end{Shaded}

\begin{verbatim}
## Analysis of Variance Table
## 
## Model 1: Output1 ~ 1 + USGG10YR + USGG2YR + USGG3M
## Model 2: Output1 ~ 1 + USGG3YR + USGG3M + USGG30YR
##   Res.Df   RSS Df Sum of Sq F Pr(>F)
## 1   8296 32.61                      
## 2   8296 53.23  0    -20.62
\end{verbatim}

Using anova, it can be seen that the model utilizing MuMIn has a smaller
residual sum of squares meaning that it is a better combination of
predictors for output 1.

Step 6 Rolling mean values for each variable.

\begin{Shaded}
\begin{Highlighting}[]
\NormalTok{Window.width<-}\DecValTok{20}\NormalTok{; Window.shift<-}\DecValTok{5}
\NormalTok{all.means<-}\KeywordTok{rollapply}\NormalTok{(AssignmentDataRegressionComparison,}\DataTypeTok{width=}\NormalTok{Window.width,}\DataTypeTok{by=}\NormalTok{Window.shift,}\DataTypeTok{by.column=}\OtherTok{TRUE}\NormalTok{, mean)}
\KeywordTok{head}\NormalTok{(all.means,}\DecValTok{10}\NormalTok{)}
\end{Highlighting}
\end{Shaded}

\begin{verbatim}
##        USGG3M  USGG6M USGG2YR USGG3YR USGG5YR USGG10YR USGG30YR  Output1
##  [1,] 15.0405 14.0855 13.2795 12.9360 12.7825  12.5780  12.1515 20.14842
##  [2,] 15.1865 14.1440 13.4855 13.1085 12.9310  12.7370  12.3370 20.55208
##  [3,] 15.2480 14.2755 13.7395 13.3390 13.1500  12.9480  12.5500 21.04895
##  [4,] 14.9345 14.0780 13.7750 13.4765 13.2385  13.0515  12.6610 21.02611
##  [5,] 14.7545 14.0585 13.9625 13.6890 13.4600  13.2295  12.8335 21.31356
##  [6,] 14.6025 14.0115 14.0380 13.7790 13.5705  13.3050  12.8890 21.39061
##  [7,] 14.0820 13.5195 13.8685 13.6710 13.4815  13.1880  12.7660 20.77200
##  [8,] 13.6255 13.0635 13.6790 13.5735 13.4270  13.1260  12.6950 20.23626
##  [9,] 13.2490 12.6810 13.5080 13.4575 13.3680  13.0770  12.6470 19.76988
## [10,] 12.9545 12.4225 13.4140 13.4240 13.3850  13.1115  12.6755 19.53054
\end{verbatim}

Above are the rolling means where the width of the windows consisted of
20 days and the shifts consisted of 5 days.

\begin{Shaded}
\begin{Highlighting}[]
\NormalTok{Count<-}\DecValTok{1}\OperatorTok{:}\KeywordTok{length}\NormalTok{(AssignmentDataRegressionComparison[,}\DecValTok{1}\NormalTok{])}
\NormalTok{Rolling.window.matrix<-}\KeywordTok{rollapply}\NormalTok{(Count,}\DataTypeTok{width=}\NormalTok{Window.width,}\DataTypeTok{by=}\NormalTok{Window.shift,}\DataTypeTok{by.column=}\OtherTok{FALSE}\NormalTok{,}
          \DataTypeTok{FUN=}\ControlFlowTok{function}\NormalTok{(z) z)}
\NormalTok{Rolling.window.matrix[}\DecValTok{1}\OperatorTok{:}\DecValTok{10}\NormalTok{,]}
\end{Highlighting}
\end{Shaded}

\begin{verbatim}
##       [,1] [,2] [,3] [,4] [,5] [,6] [,7] [,8] [,9] [,10] [,11] [,12] [,13] [,14] [,15] [,16] [,17]
##  [1,]    1    2    3    4    5    6    7    8    9    10    11    12    13    14    15    16    17
##  [2,]    6    7    8    9   10   11   12   13   14    15    16    17    18    19    20    21    22
##  [3,]   11   12   13   14   15   16   17   18   19    20    21    22    23    24    25    26    27
##  [4,]   16   17   18   19   20   21   22   23   24    25    26    27    28    29    30    31    32
##  [5,]   21   22   23   24   25   26   27   28   29    30    31    32    33    34    35    36    37
##  [6,]   26   27   28   29   30   31   32   33   34    35    36    37    38    39    40    41    42
##  [7,]   31   32   33   34   35   36   37   38   39    40    41    42    43    44    45    46    47
##  [8,]   36   37   38   39   40   41   42   43   44    45    46    47    48    49    50    51    52
##  [9,]   41   42   43   44   45   46   47   48   49    50    51    52    53    54    55    56    57
## [10,]   46   47   48   49   50   51   52   53   54    55    56    57    58    59    60    61    62
##       [,18] [,19] [,20]
##  [1,]    18    19    20
##  [2,]    23    24    25
##  [3,]    28    29    30
##  [4,]    33    34    35
##  [5,]    38    39    40
##  [6,]    43    44    45
##  [7,]    48    49    50
##  [8,]    53    54    55
##  [9,]    58    59    60
## [10,]    63    64    65
\end{verbatim}

\begin{Shaded}
\begin{Highlighting}[]
\NormalTok{Points.of.calculation<-Rolling.window.matrix[,}\DecValTok{10}\NormalTok{]}
\NormalTok{Points.of.calculation[}\DecValTok{1}\OperatorTok{:}\DecValTok{10}\NormalTok{]}
\end{Highlighting}
\end{Shaded}

\begin{verbatim}
##  [1] 10 15 20 25 30 35 40 45 50 55
\end{verbatim}

\begin{Shaded}
\begin{Highlighting}[]
\KeywordTok{length}\NormalTok{(Points.of.calculation)}
\end{Highlighting}
\end{Shaded}

\begin{verbatim}
## [1] 1657
\end{verbatim}

\begin{Shaded}
\begin{Highlighting}[]
\NormalTok{Means.forPlot<-}\KeywordTok{rep}\NormalTok{(}\OtherTok{NA}\NormalTok{,}\KeywordTok{length}\NormalTok{(AssignmentDataRegressionComparison[,}\DecValTok{1}\NormalTok{]))}
\NormalTok{Means.forPlot[Points.of.calculation]<-all.means[,}\DecValTok{1}\NormalTok{]}
\NormalTok{Means.forPlot[}\DecValTok{1}\OperatorTok{:}\DecValTok{50}\NormalTok{]}
\end{Highlighting}
\end{Shaded}

\begin{verbatim}
##  [1]      NA      NA      NA      NA      NA      NA      NA      NA      NA 15.0405      NA      NA
## [13]      NA      NA 15.1865      NA      NA      NA      NA 15.2480      NA      NA      NA      NA
## [25] 14.9345      NA      NA      NA      NA 14.7545      NA      NA      NA      NA 14.6025      NA
## [37]      NA      NA      NA 14.0820      NA      NA      NA      NA 13.6255      NA      NA      NA
## [49]      NA 13.2490
\end{verbatim}

\begin{Shaded}
\begin{Highlighting}[]
\KeywordTok{cbind}\NormalTok{(AssignmentDataRegressionComparison[,}\DecValTok{1}\NormalTok{],Means.forPlot)[}\DecValTok{1}\OperatorTok{:}\DecValTok{50}\NormalTok{,]}
\end{Highlighting}
\end{Shaded}

\begin{verbatim}
##             Means.forPlot
##  [1,] 13.52            NA
##  [2,] 13.58            NA
##  [3,] 14.50            NA
##  [4,] 14.76            NA
##  [5,] 15.20            NA
##  [6,] 15.22            NA
##  [7,] 15.24            NA
##  [8,] 15.08            NA
##  [9,] 15.25            NA
## [10,] 15.15       15.0405
## [11,] 15.79            NA
## [12,] 15.32            NA
## [13,] 15.71            NA
## [14,] 15.72            NA
## [15,] 15.70       15.1865
## [16,] 15.35            NA
## [17,] 15.20            NA
## [18,] 15.06            NA
## [19,] 14.87            NA
## [20,] 14.59       15.2480
## [21,] 14.90            NA
## [22,] 14.85            NA
## [23,] 14.67            NA
## [24,] 14.74            NA
## [25,] 15.32       14.9345
## [26,] 15.52            NA
## [27,] 15.46            NA
## [28,] 15.54            NA
## [29,] 15.51            NA
## [30,] 15.14       14.7545
## [31,] 15.02            NA
## [32,] 14.48            NA
## [33,] 14.09            NA
## [34,] 14.23            NA
## [35,] 14.15       14.6025
## [36,] 14.20            NA
## [37,] 14.14            NA
## [38,] 14.22            NA
## [39,] 14.52            NA
## [40,] 14.39       14.0820
## [41,] 14.49            NA
## [42,] 14.51            NA
## [43,] 14.29            NA
## [44,] 14.16            NA
## [45,] 13.99       13.6255
## [46,] 13.92            NA
## [47,] 13.66            NA
## [48,] 13.21            NA
## [49,] 13.02            NA
## [50,] 12.95       13.2490
\end{verbatim}

\begin{Shaded}
\begin{Highlighting}[]
\KeywordTok{plot}\NormalTok{(Means.forPlot,}\DataTypeTok{col=}\StringTok{"red"}\NormalTok{)}
\KeywordTok{lines}\NormalTok{(AssignmentDataRegressionComparison[,}\DecValTok{1}\NormalTok{])}
\end{Highlighting}
\end{Shaded}

\includegraphics{Statistical_Analysis_Project_files/figure-latex/unnamed-chunk-21-1.pdf}
It can be seen here that the means for the rolling means function seems
to fit the USGG3M data which would be expected due to the large number
of datapoints that are utilizd. The means for the rolling means function
might not fit as well if there was less data within the given dataset.

Run rolling daily difference standard deviation of each variable

\begin{Shaded}
\begin{Highlighting}[]
\NormalTok{AssignmentData.SD<-AssignmentData[,}\OperatorTok{-}\KeywordTok{c}\NormalTok{(}\DecValTok{9}\NormalTok{,}\DecValTok{10}\NormalTok{)]}
\NormalTok{AssignmentData.SD2<-AssignmentData.SD[}\OperatorTok{-}\DecValTok{1}\NormalTok{,]}
\NormalTok{AssignmentData.SD1<-AssignmentData.SD[}\DecValTok{1}\OperatorTok{:}\DecValTok{8299}\NormalTok{,]}
\NormalTok{AssignmentData.SD.Diff<-AssignmentData.SD2}\OperatorTok{-}\NormalTok{AssignmentData.SD1}
\KeywordTok{head}\NormalTok{(AssignmentData.SD.Diff,}\DecValTok{10}\NormalTok{)}
\end{Highlighting}
\end{Shaded}

\begin{verbatim}
##           USGG3M USGG6M USGG2YR USGG3YR USGG5YR USGG10YR USGG30YR     Output1
## 1/6/1981    0.06   0.07    0.14    0.03   -0.08    -0.04     0.00  0.07587222
## 1/7/1981    0.92   0.74    0.50    0.47    0.40     0.27     0.22  1.35591615
## 1/8/1981    0.26   0.10    0.17    0.17    0.07    -0.03     0.02  0.30119608
## 1/9/1981    0.44   0.30    0.44    0.33    0.20     0.22     0.22  0.82353207
## 1/12/1981   0.02  -0.07   -0.36   -0.34   -0.17    -0.12    -0.05 -0.42985741
## 1/13/1981   0.02  -0.13    0.13    0.03   -0.03     0.08     0.00  0.03935812
## 1/14/1981  -0.16  -0.20   -0.35   -0.22   -0.07     0.00    -0.01 -0.40425521
## 1/15/1981   0.17   0.19    0.30    0.27    0.16     0.09     0.18  0.52159590
## 1/16/1981  -0.10  -0.11   -0.17   -0.17   -0.11    -0.09    -0.12 -0.33130842
## 1/19/1981   0.64   0.75    0.54    0.07    0.26     0.11     0.12  0.96621424
\end{verbatim}

\begin{Shaded}
\begin{Highlighting}[]
\NormalTok{rolling.sd<-}\KeywordTok{rollapply}\NormalTok{(AssignmentData.SD.Diff,}\DataTypeTok{width=}\NormalTok{Window.width,}\DataTypeTok{by=}\NormalTok{Window.shift,}\DataTypeTok{by.column=}\OtherTok{TRUE}\NormalTok{, sd)}
\NormalTok{rolling.dates<-}\KeywordTok{rollapply}\NormalTok{(AssignmentDataRegressionComparison[}\OperatorTok{-}\DecValTok{1}\NormalTok{,],}\DataTypeTok{width=}\NormalTok{Window.width,}\DataTypeTok{by=}\NormalTok{Window.shift,}
                         \DataTypeTok{by.column=}\OtherTok{FALSE}\NormalTok{,}\DataTypeTok{FUN=}\ControlFlowTok{function}\NormalTok{(z) }\KeywordTok{rownames}\NormalTok{(z))}
\KeywordTok{head}\NormalTok{(rolling.dates)}
\end{Highlighting}
\end{Shaded}

\begin{verbatim}
##      [,1]        [,2]        [,3]        [,4]        [,5]        [,6]        [,7]       
## [1,] "1/6/1981"  "1/7/1981"  "1/8/1981"  "1/9/1981"  "1/12/1981" "1/13/1981" "1/14/1981"
## [2,] "1/13/1981" "1/14/1981" "1/15/1981" "1/16/1981" "1/19/1981" "1/20/1981" "1/21/1981"
## [3,] "1/20/1981" "1/21/1981" "1/22/1981" "1/23/1981" "1/26/1981" "1/27/1981" "1/28/1981"
## [4,] "1/27/1981" "1/28/1981" "1/29/1981" "1/30/1981" "2/2/1981"  "2/3/1981"  "2/4/1981" 
## [5,] "2/3/1981"  "2/4/1981"  "2/5/1981"  "2/6/1981"  "2/9/1981"  "2/10/1981" "2/11/1981"
## [6,] "2/10/1981" "2/11/1981" "2/13/1981" "2/17/1981" "2/18/1981" "2/19/1981" "2/20/1981"
##      [,8]        [,9]        [,10]       [,11]       [,12]       [,13]       [,14]      
## [1,] "1/15/1981" "1/16/1981" "1/19/1981" "1/20/1981" "1/21/1981" "1/22/1981" "1/23/1981"
## [2,] "1/22/1981" "1/23/1981" "1/26/1981" "1/27/1981" "1/28/1981" "1/29/1981" "1/30/1981"
## [3,] "1/29/1981" "1/30/1981" "2/2/1981"  "2/3/1981"  "2/4/1981"  "2/5/1981"  "2/6/1981" 
## [4,] "2/5/1981"  "2/6/1981"  "2/9/1981"  "2/10/1981" "2/11/1981" "2/13/1981" "2/17/1981"
## [5,] "2/13/1981" "2/17/1981" "2/18/1981" "2/19/1981" "2/20/1981" "2/23/1981" "2/24/1981"
## [6,] "2/23/1981" "2/24/1981" "2/25/1981" "2/26/1981" "2/27/1981" "3/2/1981"  "3/3/1981" 
##      [,15]       [,16]       [,17]       [,18]       [,19]       [,20]      
## [1,] "1/26/1981" "1/27/1981" "1/28/1981" "1/29/1981" "1/30/1981" "2/2/1981" 
## [2,] "2/2/1981"  "2/3/1981"  "2/4/1981"  "2/5/1981"  "2/6/1981"  "2/9/1981" 
## [3,] "2/9/1981"  "2/10/1981" "2/11/1981" "2/13/1981" "2/17/1981" "2/18/1981"
## [4,] "2/18/1981" "2/19/1981" "2/20/1981" "2/23/1981" "2/24/1981" "2/25/1981"
## [5,] "2/25/1981" "2/26/1981" "2/27/1981" "3/2/1981"  "3/3/1981"  "3/4/1981" 
## [6,] "3/4/1981"  "3/5/1981"  "3/6/1981"  "3/9/1981"  "3/10/1981" "3/11/1981"
\end{verbatim}

\begin{Shaded}
\begin{Highlighting}[]
\KeywordTok{rownames}\NormalTok{(rolling.sd)<-rolling.dates[,}\DecValTok{10}\NormalTok{]}
\KeywordTok{head}\NormalTok{(rolling.sd)}
\end{Highlighting}
\end{Shaded}

\begin{verbatim}
##              USGG3M    USGG6M   USGG2YR   USGG3YR   USGG5YR  USGG10YR  USGG30YR   Output1
## 1/19/1981 0.3433258 0.3262462 0.2748258 0.2030161 0.1713192 0.1299585 0.1202147 0.5639875
## 1/26/1981 0.2933383 0.2907504 0.2261811 0.1499219 0.1450082 0.1146895 0.1192201 0.4707427
## 2/2/1981  0.2613180 0.2437530 0.2006201 0.1632596 0.1654110 0.1459308 0.1351909 0.4681168
## 2/9/1981  0.2551754 0.2469663 0.1989446 0.1692794 0.1717219 0.1551052 0.1422183 0.4786189
## 2/18/1981 0.2480551 0.2481595 0.2102004 0.1786057 0.1744767 0.1643960 0.1516540 0.4888569
## 2/25/1981 0.1963884 0.2363672 0.2095082 0.1809180 0.1822917 0.1664956 0.1537351 0.4788897
\end{verbatim}

\begin{Shaded}
\begin{Highlighting}[]
\KeywordTok{matplot}\NormalTok{(rolling.sd[,}\KeywordTok{c}\NormalTok{(}\DecValTok{1}\NormalTok{,}\DecValTok{5}\NormalTok{,}\DecValTok{7}\NormalTok{,}\DecValTok{8}\NormalTok{)],}\DataTypeTok{xaxt=}\StringTok{"n"}\NormalTok{,}\DataTypeTok{type=}\StringTok{"l"}\NormalTok{,}\DataTypeTok{col=}\KeywordTok{c}\NormalTok{(}\StringTok{"black"}\NormalTok{,}\StringTok{"red"}\NormalTok{,}\StringTok{"blue"}\NormalTok{,}\StringTok{"green"}\NormalTok{))}
\KeywordTok{axis}\NormalTok{(}\DataTypeTok{side=}\DecValTok{1}\NormalTok{,}\DataTypeTok{at=}\DecValTok{1}\OperatorTok{:}\DecValTok{1656}\NormalTok{,}\KeywordTok{rownames}\NormalTok{(rolling.sd))}
\end{Highlighting}
\end{Shaded}

\includegraphics{Statistical_Analysis_Project_files/figure-latex/unnamed-chunk-22-1.pdf}
Q: Show periods of high volatility. How is volatility related to the
level of rates? A: The coding used to find the rolling means is also
utlized to find the rolling standard deviations. The visualization above
shows the standard deviations on a line graph of USGG3M, USGG5YR,
USGG3OYR, and Output 1. The standard deviations seem to have a sharp
rise at around 2008. This rise could be do to the precipitous drop
Output 1 has within its data set at this time period. Another
interesting inference to obtain by looking at the visualization is that
while USGG3M, USGG5YR,USGG3OYR, and Output 1 all seem to follow the same
trend, Output 1 seems to always have a higher standard deviation.

\begin{Shaded}
\begin{Highlighting}[]
\NormalTok{high.volatility.periods<-}\KeywordTok{rownames}\NormalTok{(rolling.sd)[rolling.sd[,}\DecValTok{8}\NormalTok{]}\OperatorTok{>}\NormalTok{.}\DecValTok{5}\NormalTok{]}
\NormalTok{high.volatility.periods}
\end{Highlighting}
\end{Shaded}

\begin{verbatim}
##  [1] "1/19/1981"  "3/25/1981"  "4/1/1981"   "4/8/1981"   "4/23/1981"  "4/30/1981"  "5/7/1981"  
##  [8] "5/14/1981"  "5/21/1981"  "5/29/1981"  "6/5/1981"   "6/12/1981"  "6/19/1981"  "6/26/1981" 
## [15] "7/6/1981"   "7/13/1981"  "7/20/1981"  "7/27/1981"  "10/28/1981" "11/5/1981"  "11/13/1981"
## [22] "11/20/1981" "11/30/1981" "12/7/1981"  "12/14/1981" "12/29/1981" "1/14/1982"  "1/21/1982" 
## [29] "1/28/1982"  "2/4/1982"   "2/11/1982"  "7/29/1982"  "8/5/1982"   "8/12/1982"  "8/19/1982" 
## [36] "8/26/1982"  "9/24/1982"  "10/1/1982"  "10/8/1982"  "10/18/1982" "10/13/1987" "10/20/1987"
## [43] "10/27/1987" "11/19/2007" "11/26/2007" "11/12/2008" "11/19/2008"
\end{verbatim}

\begin{Shaded}
\begin{Highlighting}[]
\NormalTok{Coefficients<-}\KeywordTok{rollapply}\NormalTok{(AssignmentDataRegressionComparison,}\DataTypeTok{width=}\NormalTok{Window.width,}\DataTypeTok{by=}\NormalTok{Window.shift,}\DataTypeTok{by.column=}\OtherTok{FALSE}\NormalTok{,}
         \DataTypeTok{FUN=}\ControlFlowTok{function}\NormalTok{(z) }\KeywordTok{coef}\NormalTok{(}\KeywordTok{lm}\NormalTok{(Output1}\OperatorTok{~}\NormalTok{USGG3M}\OperatorTok{+}\NormalTok{USGG5YR}\OperatorTok{+}\NormalTok{USGG30YR,}\DataTypeTok{data=}\KeywordTok{as.data.frame}\NormalTok{(z))))}
\NormalTok{rolling.dates<-}\KeywordTok{rollapply}\NormalTok{(AssignmentDataRegressionComparison[,}\DecValTok{1}\OperatorTok{:}\DecValTok{8}\NormalTok{],}\DataTypeTok{width=}\NormalTok{Window.width,}\DataTypeTok{by=}\NormalTok{Window.shift,}\DataTypeTok{by.column=}\OtherTok{FALSE}\NormalTok{,}
                         \DataTypeTok{FUN=}\ControlFlowTok{function}\NormalTok{(z) }\KeywordTok{rownames}\NormalTok{(z))}

\KeywordTok{rownames}\NormalTok{(Coefficients)<-rolling.dates[,}\DecValTok{10}\NormalTok{]}
\NormalTok{Coefficients[}\DecValTok{1}\OperatorTok{:}\DecValTok{10}\NormalTok{,]}
\end{Highlighting}
\end{Shaded}

\begin{verbatim}
##           (Intercept)    USGG3M  USGG5YR  USGG30YR
## 1/16/1981   -15.70877 0.6723609 1.797680 0.2276011
## 1/23/1981   -15.96684 0.6948992 1.480514 0.5529139
## 1/30/1981   -16.77273 0.7078197 1.434388 0.6507280
## 2/6/1981    -16.90734 0.7279033 1.470083 0.6003377
## 2/17/1981   -17.46896 0.7343406 1.361289 0.7499705
## 2/24/1981   -17.04722 0.7357663 1.295641 0.7844908
## 3/3/1981    -17.67901 0.8544681 1.396653 0.5945022
## 3/10/1981   -17.72402 0.9162385 1.654274 0.2571200
## 3/17/1981   -17.00260 0.9265767 1.647852 0.1951273
## 3/24/1981   -16.84302 0.9102780 1.477727 0.3788401
\end{verbatim}

\begin{Shaded}
\begin{Highlighting}[]
\KeywordTok{pairs}\NormalTok{(Coefficients)}
\end{Highlighting}
\end{Shaded}

\includegraphics{Statistical_Analysis_Project_files/figure-latex/unnamed-chunk-23-1.pdf}

\begin{Shaded}
\begin{Highlighting}[]
\KeywordTok{matplot}\NormalTok{(Coefficients[,}\OperatorTok{-}\DecValTok{1}\NormalTok{],}\DataTypeTok{xaxt=}\StringTok{"n"}\NormalTok{,}\DataTypeTok{type=}\StringTok{"l"}\NormalTok{,}\DataTypeTok{col=}\KeywordTok{c}\NormalTok{(}\StringTok{"black"}\NormalTok{,}\StringTok{"red"}\NormalTok{,}\StringTok{"green"}\NormalTok{))}
\KeywordTok{axis}\NormalTok{(}\DataTypeTok{side=}\DecValTok{1}\NormalTok{,}\DataTypeTok{at=}\DecValTok{1}\OperatorTok{:}\DecValTok{1657}\NormalTok{,}\KeywordTok{rownames}\NormalTok{(Coefficients))}
\end{Highlighting}
\end{Shaded}

\includegraphics{Statistical_Analysis_Project_files/figure-latex/unnamed-chunk-23-2.pdf}
Pairs plot\\
The pairs plot of coefficients represents the coefficients between a
short term bond(USGG3M), a medium term bond(USGG5YR), and a long term
bond(USGG30YR). The medium and long term bonds share a stong negative
correlation with respect to their coefficients. This means that their
yields could be attempting to counteract one another throughout the
dataset. The short term bond doesnt seem to share much correlation with
respect to it's coefficients with the medium and long term bonds. The
intercept doesn't seem to share much correlation with the short term or
medium term bonds, but it does seem to have a slight negative
correlation with the long term bond. The third visualization looks at
the coefficients for the short, medium and long term bonds. The
coefficients seem to diverge at times from their means. The dates where
this phenomon can be seen are within the years of 1987,1991, 2005 and
2012.

Q: Is the picture of coefficients consistent with the picture of pairs?
If yes, explain why. A: Yes, it was stated that the 30 yr and 5 yr
shared a stronger netagtive correlation with one another, and neither
seemed to correlate with the 3 months. It can be seen that the green
line(Long term bond) and the red line (medium term bond) seem to reflect
one another. The black line remains between the two for a majority of
the visualization.

\begin{Shaded}
\begin{Highlighting}[]
\NormalTok{high.slopespread.periods<-}\KeywordTok{rownames}\NormalTok{(Coefficients)[Coefficients[,}\DecValTok{3}\NormalTok{]}\OperatorTok{-}\NormalTok{Coefficients[,}\DecValTok{4}\NormalTok{]}\OperatorTok{>}\DecValTok{3}\NormalTok{]}
\NormalTok{jump.slopes<-}\KeywordTok{rownames}\NormalTok{(Coefficients)[Coefficients[,}\DecValTok{3}\NormalTok{]}\OperatorTok{>}\DecValTok{3}\NormalTok{]}
\NormalTok{high.slopespread.periods}
\end{Highlighting}
\end{Shaded}

\begin{verbatim}
##  [1] "4/26/1982"  "12/15/1982" "9/16/1985"  "5/12/1987"  "5/19/1987"  "5/27/1987"  "9/25/1987" 
##  [8] "3/15/1988"  "9/27/1988"  "10/5/1988"  "3/10/1989"  "2/5/1992"   "8/3/1994"   "12/8/1994" 
## [15] "6/14/1996"  "5/9/1997"   "5/16/1997"  "5/23/1997"  "5/30/1997"  "12/26/2000" "1/2/2001"  
## [22] "7/25/2001"  "8/1/2001"   "11/13/2003" "8/12/2004"  "12/16/2004"
\end{verbatim}

This shows when dates where USGG5YR had a greater coefficeint than
USGG30YR.This would be where the green line goes above the red line in
the previous visualizaiton.

\begin{Shaded}
\begin{Highlighting}[]
\NormalTok{r.squared<-}\KeywordTok{rollapply}\NormalTok{(AssignmentDataRegressionComparison,}\DataTypeTok{width=}\NormalTok{Window.width,}\DataTypeTok{by=}\NormalTok{Window.shift,}\DataTypeTok{by.column=}\OtherTok{FALSE}\NormalTok{,}
         \DataTypeTok{FUN=}\ControlFlowTok{function}\NormalTok{(z) }\KeywordTok{summary}\NormalTok{(}\KeywordTok{lm}\NormalTok{(Output1}\OperatorTok{~}\NormalTok{USGG3M}\OperatorTok{+}\NormalTok{USGG5YR}\OperatorTok{+}\NormalTok{USGG30YR,}\DataTypeTok{data=}\KeywordTok{as.data.frame}\NormalTok{(z)))}\OperatorTok{$}\NormalTok{r.squared)}
\NormalTok{r.squared<-}\KeywordTok{cbind}\NormalTok{(rolling.dates[,}\DecValTok{10}\NormalTok{],r.squared)}
\NormalTok{r.squared[}\DecValTok{1}\OperatorTok{:}\DecValTok{10}\NormalTok{,]}
\end{Highlighting}
\end{Shaded}

\begin{verbatim}
##                   r.squared          
##  [1,] "1/16/1981" "0.995046300986446"
##  [2,] "1/23/1981" "0.992485868136646"
##  [3,] "1/30/1981" "0.998641209587999"
##  [4,] "2/6/1981"  "0.998849080081881"
##  [5,] "2/17/1981" "0.997958757207598"
##  [6,] "2/24/1981" "0.996489757136839"
##  [7,] "3/3/1981"  "0.99779753570421" 
##  [8,] "3/10/1981" "0.998963395226792"
##  [9,] "3/17/1981" "0.998729445388789"
## [10,] "3/24/1981" "0.997073000898673"
\end{verbatim}

\begin{Shaded}
\begin{Highlighting}[]
\KeywordTok{plot}\NormalTok{(r.squared[,}\DecValTok{2}\NormalTok{],}\DataTypeTok{xaxt=}\StringTok{"n"}\NormalTok{,}\DataTypeTok{ylim=}\KeywordTok{c}\NormalTok{(}\DecValTok{0}\NormalTok{,}\DecValTok{1}\NormalTok{))}
\KeywordTok{axis}\NormalTok{(}\DataTypeTok{side=}\DecValTok{1}\NormalTok{,}\DataTypeTok{at=}\DecValTok{1}\OperatorTok{:}\DecValTok{1657}\NormalTok{,}\KeywordTok{rownames}\NormalTok{(Coefficients))}
\end{Highlighting}
\end{Shaded}

\includegraphics{Statistical_Analysis_Project_files/figure-latex/unnamed-chunk-25-1.pdf}
Q:How often the R-Squared is not conidered high? All of the R-squared
seem to be very high. They will not share that level of correlation when
their coffeicients diverge from their average which was spoken to with
the last visualization. In the second visualization, the R-squared is
near 100\% for a majority of the data set. It does diverge a bit lower
than 95\%, and this can be attributed to the fact that the coefficients
diverge from their means at certain points of the data set which can be
seen in the previous visualization. The dates where this phenomon can be
seen are within the years of 1987,1991, 2005 and 2012. This is
determined through the code below.

\begin{Shaded}
\begin{Highlighting}[]
\NormalTok{(low.r.squared.periods<-r.squared[r.squared[,}\DecValTok{2}\NormalTok{]}\OperatorTok{<}\NormalTok{.}\DecValTok{9}\NormalTok{,}\DecValTok{1}\NormalTok{])}
\end{Highlighting}
\end{Shaded}

\begin{verbatim}
## [1] "6/24/1987" "6/27/1991" "4/28/2005" "6/20/2012"
\end{verbatim}

Q: What could cause the decrease of R2: A: There are certain dates wehre
the data for the coeffieicnets extend past one another. With the plot of
coefficients above, the extension of these models can be seen which
would cause for low R2. The dates where this phenomon can be seen are
within the years of 1987,1991, 2005 and 2012.

\begin{Shaded}
\begin{Highlighting}[]
\NormalTok{Pvalues<-}\KeywordTok{rollapply}\NormalTok{(AssignmentDataRegressionComparison,}\DataTypeTok{width=}\NormalTok{Window.width,}\DataTypeTok{by=}\NormalTok{Window.shift,}\DataTypeTok{by.column=}\OtherTok{FALSE}\NormalTok{,}
                        \DataTypeTok{FUN=}\ControlFlowTok{function}\NormalTok{(z) }\KeywordTok{summary}\NormalTok{(}\KeywordTok{lm}\NormalTok{(Output1}\OperatorTok{~}\NormalTok{USGG3M}\OperatorTok{+}\NormalTok{USGG5YR}\OperatorTok{+}\NormalTok{USGG30YR,}\DataTypeTok{data=}\KeywordTok{as.data.frame}\NormalTok{(z)))}\OperatorTok{$}\NormalTok{coefficients[,}\DecValTok{4}\NormalTok{])}
\KeywordTok{rownames}\NormalTok{(Pvalues)<-rolling.dates[,}\DecValTok{10}\NormalTok{]}
\NormalTok{Pvalues[}\DecValTok{1}\OperatorTok{:}\DecValTok{10}\NormalTok{,]}
\end{Highlighting}
\end{Shaded}

\begin{verbatim}
##            (Intercept)       USGG3M      USGG5YR     USGG30YR
## 1/16/1981 1.193499e-10 3.764585e-10 2.391260e-07 2.538852e-01
## 1/23/1981 3.751077e-12 1.008053e-11 2.447369e-07 5.300949e-03
## 1/30/1981 3.106359e-18 1.406387e-14 4.040035e-09 3.626961e-05
## 2/6/1981  2.591522e-19 3.360104e-19 3.828054e-11 2.221691e-05
## 2/17/1981 1.897239e-16 6.578118e-17 1.461743e-09 1.331767e-04
## 2/24/1981 2.341158e-13 1.000212e-13 9.008221e-07 4.733543e-03
## 3/3/1981  5.435581e-14 1.535503e-11 3.357199e-06 6.010473e-02
## 3/10/1981 6.227624e-16 1.178498e-16 1.679479e-05 3.851840e-01
## 3/17/1981 9.592582e-17 7.065226e-20 1.459692e-05 5.025726e-01
## 3/24/1981 8.248747e-16 6.689840e-16 6.413371e-04 3.052705e-01
\end{verbatim}

\begin{Shaded}
\begin{Highlighting}[]
\KeywordTok{matplot}\NormalTok{(Pvalues,}\DataTypeTok{xaxt=}\StringTok{"n"}\NormalTok{,}\DataTypeTok{col=}\KeywordTok{c}\NormalTok{(}\StringTok{"black"}\NormalTok{,}\StringTok{"blue"}\NormalTok{,}\StringTok{"red"}\NormalTok{,}\StringTok{"green"}\NormalTok{),}\DataTypeTok{type=}\StringTok{"o"}\NormalTok{)}
\KeywordTok{axis}\NormalTok{(}\DataTypeTok{side=}\DecValTok{1}\NormalTok{,}\DataTypeTok{at=}\DecValTok{1}\OperatorTok{:}\DecValTok{1657}\NormalTok{,}\KeywordTok{rownames}\NormalTok{(Coefficients))}
\end{Highlighting}
\end{Shaded}

\includegraphics{Statistical_Analysis_Project_files/figure-latex/unnamed-chunk-27-1.pdf}

What does this graph tell us? The USGG30YR bonds were non-significant
against the output up until 2008. After 2008, the USGG3M began to have
strong insignificance. The USGG5YR has a short period of insiginificance
at the end of the 80s.

Below, the dates of insignificance can be observed for the USGG3M,
USGG\%YR, and USGG30YR can be seen.

\begin{Shaded}
\begin{Highlighting}[]
\KeywordTok{c}\NormalTok{(}\StringTok{"USGG3M"}\NormalTok{)}
\end{Highlighting}
\end{Shaded}

\begin{verbatim}
## [1] "USGG3M"
\end{verbatim}

\begin{Shaded}
\begin{Highlighting}[]
\KeywordTok{rownames}\NormalTok{(Pvalues)[Pvalues[,}\DecValTok{2}\NormalTok{]}\OperatorTok{>}\NormalTok{.}\DecValTok{5}\NormalTok{]}
\end{Highlighting}
\end{Shaded}

\begin{verbatim}
##  [1] "7/15/1992"  "7/26/1996"  "8/2/1996"   "11/7/2000"  "5/30/2001"  "5/2/2002"   "5/16/2002" 
##  [8] "5/23/2002"  "1/30/2003"  "2/6/2003"   "7/24/2003"  "7/31/2003"  "8/7/2003"   "11/20/2003"
## [15] "12/18/2003" "4/28/2005"  "2/10/2006"  "3/9/2007"   "3/16/2007"  "7/21/2009"  "10/6/2009" 
## [22] "10/13/2009" "12/28/2010" "1/11/2011"  "3/1/2011"   "11/16/2011" "11/23/2011" "5/23/2012" 
## [29] "7/11/2012"  "6/6/2013"   "1/16/2014"  "1/30/2014"  "3/6/2014"
\end{verbatim}

\begin{Shaded}
\begin{Highlighting}[]
\KeywordTok{c}\NormalTok{(}\StringTok{"USGG5YR"}\NormalTok{)}
\end{Highlighting}
\end{Shaded}

\begin{verbatim}
## [1] "USGG5YR"
\end{verbatim}

\begin{Shaded}
\begin{Highlighting}[]
\KeywordTok{rownames}\NormalTok{(Pvalues)[Pvalues[,}\DecValTok{3}\NormalTok{]}\OperatorTok{>}\NormalTok{.}\DecValTok{5}\NormalTok{]}
\end{Highlighting}
\end{Shaded}

\begin{verbatim}
## [1] "12/1/1982" "3/16/1987" "4/28/1987" "6/24/1987" "9/3/1987"  "9/11/1987" "9/20/1988" "12/3/1999"
\end{verbatim}

\begin{Shaded}
\begin{Highlighting}[]
\KeywordTok{c}\NormalTok{(}\StringTok{"USGG30YR"}\NormalTok{)}
\end{Highlighting}
\end{Shaded}

\begin{verbatim}
## [1] "USGG30YR"
\end{verbatim}

\begin{Shaded}
\begin{Highlighting}[]
\KeywordTok{rownames}\NormalTok{(Pvalues)[Pvalues[,}\DecValTok{4}\NormalTok{]}\OperatorTok{>}\NormalTok{.}\DecValTok{5}\NormalTok{]}
\end{Highlighting}
\end{Shaded}

\begin{verbatim}
##   [1] "3/17/1981"  "4/22/1981"  "4/29/1981"  "6/4/1981"   "10/13/1981" "11/19/1981" "2/3/1982"  
##   [8] "2/26/1982"  "4/2/1982"   "4/12/1982"  "5/3/1982"   "7/7/1982"   "9/1/1982"   "9/23/1982" 
##  [15] "12/8/1982"  "2/18/1983"  "3/1/1983"   "5/4/1983"   "12/30/1983" "1/9/1984"   "2/6/1984"  
##  [22] "3/14/1984"  "3/21/1984"  "4/11/1984"  "6/22/1984"  "6/29/1984"  "10/31/1984" "11/16/1984"
##  [29] "11/26/1984" "12/17/1984" "3/25/1985"  "5/14/1985"  "5/21/1985"  "8/30/1985"  "9/9/1985"  
##  [36] "9/23/1985"  "10/1/1985"  "10/8/1985"  "10/16/1985" "10/23/1985" "10/30/1985" "11/14/1985"
##  [43] "11/21/1985" "1/22/1986"  "1/29/1986"  "5/5/1987"   "12/16/1987" "1/25/1988"  "2/1/1988"  
##  [50] "2/16/1988"  "3/1/1988"   "3/22/1988"  "5/18/1988"  "6/3/1988"   "6/10/1988"  "6/24/1988" 
##  [57] "7/25/1988"  "8/15/1988"  "12/5/1988"  "2/2/1989"   "3/3/1989"   "4/10/1989"  "5/1/1989"  
##  [64] "6/13/1989"  "8/16/1989"  "9/14/1989"  "9/21/1989"  "10/3/1989"  "10/11/1989" "10/18/1989"
##  [71] "11/1/1989"  "11/30/1989" "12/7/1989"  "1/8/1990"   "1/16/1990"  "6/15/1990"  "7/30/1990" 
##  [78] "8/6/1990"   "10/2/1990"  "10/10/1990" "11/23/1990" "3/1/1991"   "5/13/1991"  "5/20/1991" 
##  [85] "6/13/1991"  "7/5/1991"   "7/19/1991"  "9/16/1991"  "2/12/1992"  "2/20/1992"  "3/12/1992" 
##  [92] "4/16/1992"  "4/24/1992"  "5/1/1992"   "5/8/1992"   "6/2/1992"   "6/9/1992"   "6/16/1992" 
##  [99] "8/19/1992"  "8/26/1992"  "10/23/1992" "5/20/1993"  "6/11/1993"  "6/18/1993"  "8/30/1993" 
## [106] "12/15/1993" "12/22/1993" "3/16/1994"  "3/30/1994"  "4/6/1994"   "4/20/1994"  "4/27/1994" 
## [113] "6/29/1994"  "8/17/1994"  "9/21/1994"  "9/28/1994"  "12/22/1994" "12/29/1994" "1/5/1995"  
## [120] "1/12/1995"  "1/26/1995"  "2/2/1995"   "2/9/1995"   "4/6/1995"   "4/13/1995"  "8/25/1995" 
## [127] "9/29/1995"  "10/27/1995" "11/3/1995"  "11/10/1995" "12/29/1995" "1/5/1996"   "1/12/1996" 
## [134] "1/19/1996"  "3/29/1996"  "5/31/1996"  "6/21/1996"  "7/12/1996"  "7/19/1996"  "7/26/1996" 
## [141] "8/2/1996"   "8/9/1996"   "8/16/1996"  "8/23/1996"  "9/27/1996"  "10/4/1996"  "12/6/1996" 
## [148] "2/28/1997"  "3/7/1997"   "4/18/1997"  "4/25/1997"  "5/2/1997"   "6/13/1997"  "6/20/1997" 
## [155] "6/27/1997"  "7/4/1997"   "10/10/1997" "10/17/1997" "12/12/1997" "12/19/1997" "12/26/1997"
## [162] "1/9/1998"   "1/16/1998"  "8/14/1998"  "8/21/1998"  "8/28/1998"  "9/18/1998"  "9/25/1998" 
## [169] "12/4/1998"  "12/11/1998" "1/8/1999"   "3/12/1999"  "4/2/1999"   "5/14/1999"  "6/25/1999" 
## [176] "7/9/1999"   "7/30/1999"  "8/20/1999"  "9/10/1999"  "9/24/1999"  "10/15/1999" "12/31/1999"
## [183] "3/3/2000"   "3/31/2000"  "4/7/2000"   "4/14/2000"  "4/21/2000"  "7/25/2000"  "11/21/2000"
## [190] "11/28/2000" "3/7/2001"   "5/30/2001"  "7/11/2001"  "10/11/2001" "12/13/2001" "1/24/2002" 
## [197] "1/31/2002"  "8/29/2002"  "9/26/2002"  "12/26/2002" "1/16/2003"  "1/23/2003"  "3/6/2003"  
## [204] "3/13/2003"  "3/20/2003"  "3/27/2003"  "5/1/2003"   "6/19/2003"  "6/26/2003"  "7/3/2003"  
## [211] "9/18/2003"  "9/25/2003"  "10/16/2003" "10/23/2003" "10/30/2003" "11/20/2003" "1/1/2004"  
## [218] "2/5/2004"   "2/26/2004"  "3/4/2004"   "4/15/2004"  "4/22/2004"  "5/13/2004"  "5/27/2004" 
## [225] "6/3/2004"   "6/17/2004"  "6/24/2004"  "7/8/2004"   "10/14/2004" "10/21/2004" "10/28/2004"
## [232] "11/18/2004" "12/23/2004" "3/17/2005"  "3/24/2005"  "3/31/2005"  "4/7/2005"   "7/21/2005" 
## [239] "8/11/2005"  "9/22/2005"  "10/6/2005"  "11/3/2005"  "12/8/2005"  "12/15/2005" "1/20/2006" 
## [246] "5/12/2006"  "5/19/2006"  "5/26/2006"  "6/2/2006"   "6/9/2006"   "6/16/2006"  "7/7/2006"  
## [253] "7/21/2006"  "10/6/2006"  "11/3/2006"  "1/12/2007"  "2/23/2007"  "3/16/2007"  "4/27/2007" 
## [260] "5/25/2007"  "9/7/2007"   "9/14/2007"  "9/21/2007"  "9/28/2007"  "11/16/2007" "5/5/2009"  
## [267] "6/1/2010"   "6/15/2010"  "1/25/2011"  "2/1/2011"   "6/12/2014"
\end{verbatim}

Step 7

\begin{Shaded}
\begin{Highlighting}[]
\NormalTok{AssignmentData.Output<-(AssignmentData}\OperatorTok{$}\NormalTok{Output1)}
\NormalTok{AssignmentData<-}\KeywordTok{data.matrix}\NormalTok{(AssignmentData[,}\DecValTok{1}\OperatorTok{:}\DecValTok{7}\NormalTok{],}\DataTypeTok{rownames.force=}\StringTok{"automatic"}\NormalTok{)}
\NormalTok{AssignmentData.cov<-}\KeywordTok{data.matrix}\NormalTok{(AssignmentData[,}\DecValTok{1}\OperatorTok{:}\DecValTok{7}\NormalTok{], }\DataTypeTok{rownames.force=}\StringTok{"NA"}\NormalTok{)}
\KeywordTok{dim}\NormalTok{(AssignmentData)}
\end{Highlighting}
\end{Shaded}

\begin{verbatim}
## [1] 8300    7
\end{verbatim}

\begin{Shaded}
\begin{Highlighting}[]
\KeywordTok{head}\NormalTok{(AssignmentData)}
\end{Highlighting}
\end{Shaded}

\begin{verbatim}
##           USGG3M USGG6M USGG2YR USGG3YR USGG5YR USGG10YR USGG30YR
## 1/5/1981   13.52  13.09  12.289   12.28  12.294   12.152   11.672
## 1/6/1981   13.58  13.16  12.429   12.31  12.214   12.112   11.672
## 1/7/1981   14.50  13.90  12.929   12.78  12.614   12.382   11.892
## 1/8/1981   14.76  14.00  13.099   12.95  12.684   12.352   11.912
## 1/9/1981   15.20  14.30  13.539   13.28  12.884   12.572   12.132
## 1/12/1981  15.22  14.23  13.179   12.94  12.714   12.452   12.082
\end{verbatim}

Explore the dimensionality of the set of 3, 2Y and 5Y yields.

\begin{Shaded}
\begin{Highlighting}[]
\NormalTok{AssignmentData.3M_2Y_5Y<-AssignmentData[,}\KeywordTok{c}\NormalTok{(}\DecValTok{1}\NormalTok{,}\DecValTok{3}\NormalTok{,}\DecValTok{5}\NormalTok{)]}
\KeywordTok{length}\NormalTok{(AssignmentData.3M_2Y_5Y)}
\end{Highlighting}
\end{Shaded}

\begin{verbatim}
## [1] 24900
\end{verbatim}

\begin{Shaded}
\begin{Highlighting}[]
\KeywordTok{pairs}\NormalTok{(AssignmentData.3M_2Y_5Y)}
\end{Highlighting}
\end{Shaded}

\includegraphics{Statistical_Analysis_Project_files/figure-latex/unnamed-chunk-30-1.pdf}

\begin{Shaded}
\begin{Highlighting}[]
\KeywordTok{cov}\NormalTok{(AssignmentData[,}\OperatorTok{-}\KeywordTok{c}\NormalTok{(}\DecValTok{8}\NormalTok{,}\DecValTok{9}\NormalTok{)])}
\end{Highlighting}
\end{Shaded}

\begin{verbatim}
##             USGG3M    USGG6M   USGG2YR   USGG3YR   USGG5YR  USGG10YR USGG30YR
## USGG3M   11.760393 11.855287 12.303031 11.942035 11.188856  9.924865 8.587987
## USGG6M   11.855287 12.000510 12.512434 12.158422 11.406959 10.128890 8.768702
## USGG2YR  12.303031 12.512434 13.284203 12.977542 12.279514 11.005377 9.600181
## USGG3YR  11.942035 12.158422 12.977542 12.708647 12.068078 10.856033 9.497246
## USGG5YR  11.188856 11.406959 12.279514 12.068078 11.543082 10.463386 9.212159
## USGG10YR  9.924865 10.128890 11.005377 10.856033 10.463386  9.583483 8.510632
## USGG30YR  8.587987  8.768702  9.600181  9.497246  9.212159  8.510632 7.624304
\end{verbatim}

The covariance matrix's values initially increase as you move to the
bottom right of the matrix, but then they begin to decrease. This means
that USGG2YR, USGG3YR, and USGG5YR share a strong amount of variance
within their data sets with other short and mid term bonds. The longer
term bonds share less variance with one another. The USGG2YR and the
USGG5YR share a very strong positive correlation wtih one anotehr. The
USGG3M also shares a strong polsitive correlation with the two otehr
predictors but there seems to be some deviation from that correlation
towards the end of each line. The deviations for the USGG3M could be due
to the insignificance of the coefficient towards the end of the data set
as seen previously in the visualization which looked at the dates of
insignificance for USGG3M.

Observe the 3d plot of the set.

\begin{Shaded}
\begin{Highlighting}[]
\KeywordTok{rgl.points}\NormalTok{(AssignmentData.3M_2Y_5Y)}
\end{Highlighting}
\end{Shaded}

The three dimensional plot produced through the code above combines the
data sets for the USGG3M, USGG2YR, and the USGG5YR data. It is flat due
to the fact that USGG5YR and USGG2YR share such a strong correlation. We
can see deviations which reflect USGG3M in the pair plot above which is
representative of the deviation of USGG3M to both USGG2YR and USGG5YR.

Manually Solving for Covariance

\begin{Shaded}
\begin{Highlighting}[]
\NormalTok{AssignmentData.means<-}\KeywordTok{cbind}\NormalTok{(}\KeywordTok{rep}\NormalTok{(}\KeywordTok{mean}\NormalTok{(AssignmentData[,}\DecValTok{1}\NormalTok{]),}\DecValTok{8300}\NormalTok{),}\KeywordTok{rep}\NormalTok{(}\KeywordTok{mean}\NormalTok{(AssignmentData[,}\DecValTok{2}\NormalTok{]),}\DecValTok{8300}\NormalTok{),}\KeywordTok{rep}\NormalTok{(}\KeywordTok{mean}\NormalTok{(AssignmentData[,}\DecValTok{3}\NormalTok{]),}\DecValTok{8300}\NormalTok{),}\KeywordTok{rep}\NormalTok{(}\KeywordTok{mean}\NormalTok{(AssignmentData[,}\DecValTok{4}\NormalTok{]),}\DecValTok{8300}\NormalTok{),}\KeywordTok{rep}\NormalTok{(}\KeywordTok{mean}\NormalTok{(AssignmentData[,}\DecValTok{5}\NormalTok{]),}\DecValTok{8300}\NormalTok{),}\KeywordTok{rep}\NormalTok{(}\KeywordTok{mean}\NormalTok{(AssignmentData[,}\DecValTok{6}\NormalTok{]),}\DecValTok{8300}\NormalTok{),}\KeywordTok{rep}\NormalTok{(}\KeywordTok{mean}\NormalTok{(AssignmentData[,}\DecValTok{7}\NormalTok{]),}\DecValTok{8300}\NormalTok{))}
\NormalTok{D<-AssignmentData.cov}\OperatorTok{-}\NormalTok{AssignmentData.means}
\NormalTok{D<-}\KeywordTok{as.matrix}\NormalTok{(D)}
\NormalTok{n<-}\KeywordTok{nrow}\NormalTok{(AssignmentData)}
\NormalTok{C<-(}\KeywordTok{t}\NormalTok{(D)}\OperatorTok\NormalTok{(D))}\OperatorTok{/}\NormalTok{n}
\NormalTok{C}
\end{Highlighting}
\end{Shaded}

\begin{verbatim}
##             USGG3M    USGG6M   USGG2YR   USGG3YR   USGG5YR  USGG10YR USGG30YR
## USGG3M   11.758976 11.853859 12.301548 11.940597 11.187507  9.923669 8.586952
## USGG6M   11.853859 11.999064 12.510927 12.156957 11.405584 10.127670 8.767646
## USGG2YR  12.301548 12.510927 13.282602 12.975978 12.278035 11.004051 9.599024
## USGG3YR  11.940597 12.156957 12.975978 12.707115 12.066624 10.854725 9.496102
## USGG5YR  11.187507 11.405584 12.278035 12.066624 11.541691 10.462125 9.211049
## USGG10YR  9.923669 10.127670 11.004051 10.854725 10.462125  9.582328 8.509607
## USGG30YR  8.586952  8.767646  9.599024  9.496102  9.211049  8.509607 7.623386
\end{verbatim}

In object D, the data is centered by subtracting the mean. By taking the
dot product between the transpose of D and D and then diving by n, the
covariance matrix is produced.

\begin{Shaded}
\begin{Highlighting}[]
\NormalTok{Covariance.Matrix<-}\KeywordTok{cov}\NormalTok{(AssignmentData[,}\OperatorTok{-}\KeywordTok{c}\NormalTok{(}\DecValTok{8}\NormalTok{,}\DecValTok{9}\NormalTok{)])}
\NormalTok{Covariance.Matrix}
\end{Highlighting}
\end{Shaded}

\begin{verbatim}
##             USGG3M    USGG6M   USGG2YR   USGG3YR   USGG5YR  USGG10YR USGG30YR
## USGG3M   11.760393 11.855287 12.303031 11.942035 11.188856  9.924865 8.587987
## USGG6M   11.855287 12.000510 12.512434 12.158422 11.406959 10.128890 8.768702
## USGG2YR  12.303031 12.512434 13.284203 12.977542 12.279514 11.005377 9.600181
## USGG3YR  11.942035 12.158422 12.977542 12.708647 12.068078 10.856033 9.497246
## USGG5YR  11.188856 11.406959 12.279514 12.068078 11.543082 10.463386 9.212159
## USGG10YR  9.924865 10.128890 11.005377 10.856033 10.463386  9.583483 8.510632
## USGG30YR  8.587987  8.768702  9.600181  9.497246  9.212159  8.510632 7.624304
\end{verbatim}

\begin{Shaded}
\begin{Highlighting}[]
\NormalTok{Maturities<-}\KeywordTok{c}\NormalTok{(.}\DecValTok{25}\NormalTok{,.}\DecValTok{5}\NormalTok{,}\DecValTok{2}\NormalTok{,}\DecValTok{3}\NormalTok{,}\DecValTok{5}\NormalTok{,}\DecValTok{10}\NormalTok{,}\DecValTok{30}\NormalTok{)}
\KeywordTok{contour}\NormalTok{(Maturities,Maturities,Covariance.Matrix)}
\end{Highlighting}
\end{Shaded}

\includegraphics{Statistical_Analysis_Project_files/figure-latex/unnamed-chunk-33-1.pdf}
Cov() can also be utilized to create the covariance matrix. The countour
plot attempts to create a 3d image using a 2d image. It can be seen that
the lower left part of the graph has the highest elements of the graph
with a height of 12.5. This number gradually decreases as we move to the
top right of the graph. This means that the longer bonds tend to share
less covariance with other bonds than the shorter term bonds. This would
make sense becuase predicting the market in the long run is incredibly
difficult therefore rates shouldn't fluctuate nearly as much as they
would with the short term bonds.

PCA Manually Done

\begin{Shaded}
\begin{Highlighting}[]
\NormalTok{AssignmentData.Centered<-AssignmentData}\OperatorTok{-}\NormalTok{AssignmentData.means}
\NormalTok{Eigen.Decomposition<-}\KeywordTok{eigen}\NormalTok{(Covariance.Matrix)}
\NormalTok{Loadings <-}\StringTok{ }\NormalTok{Eigen.Decomposition}\OperatorTok{$}\NormalTok{vectors}
\NormalTok{Factors <-}\StringTok{ }\NormalTok{AssignmentData.Centered}\OperatorTok\NormalTok{Loadings}
\NormalTok{AssignmentData.Output.Factors<-}\KeywordTok{as.data.frame}\NormalTok{(}\KeywordTok{cbind}\NormalTok{(AssignmentData.Output,Factors))}
\NormalTok{LinMod.PCA<-}\KeywordTok{lm}\NormalTok{(AssignmentData.Output}\OperatorTok{~}\NormalTok{.,AssignmentData.Output.Factors)}
\KeywordTok{calc.relimp}\NormalTok{(LinMod.PCA)}
\end{Highlighting}
\end{Shaded}

\begin{verbatim}
## Response variable: AssignmentData.Output 
## Total response variance: 76.80444 
## Analysis based on 8300 observations 
## 
## 7 Regressors: 
## V2 V3 V4 V5 V6 V7 V8 
## Proportion of variance explained by model: 100%
## Metrics are not normalized (rela=FALSE). 
## 
## Relative importance metrics: 
## 
##    lmg
## V2   1
## V3   0
## V4   0
## V5   0
## V6   0
## V7   0
## V8   0
## 
## Average coefficients for different model sizes: 
## 
##               1X           2Xs           3Xs           4Xs           5Xs           6Xs
## V2 -1.000000e+00 -1.000000e+00 -1.000000e+00 -1.000000e+00 -1.000000e+00 -1.000000e+00
## V3 -8.736166e-12 -8.738461e-12 -8.740756e-12 -8.743050e-12 -8.745345e-12 -8.747640e-12
## V4  4.488087e-11  4.489814e-11  4.491541e-11  4.493269e-11  4.494996e-11  4.496723e-11
## V5 -1.077551e-11 -1.060446e-11 -1.043341e-11 -1.026236e-11 -1.009131e-11 -9.920260e-12
## V6  2.189490e-10  2.190589e-10  2.191688e-10  2.192786e-10  2.193885e-10  2.194984e-10
## V7 -2.103928e-10 -2.099526e-10 -2.095124e-10 -2.090722e-10 -2.086320e-10 -2.081917e-10
## V8  3.369576e-10  3.377625e-10  3.385675e-10  3.393725e-10  3.401774e-10  3.409824e-10
##              7Xs
## V2 -1.000000e+00
## V3 -8.749935e-12
## V4  4.498450e-11
## V5 -9.749210e-12
## V6  2.196083e-10
## V7 -2.077515e-10
## V8  3.417874e-10
\end{verbatim}

\begin{Shaded}
\begin{Highlighting}[]
\KeywordTok{barplot}\NormalTok{(Eigen.Decomposition}\OperatorTok{$}\NormalTok{values}\OperatorTok{/}\KeywordTok{sum}\NormalTok{(Eigen.Decomposition}\OperatorTok{$}\NormalTok{values),}\DataTypeTok{width=}\DecValTok{2}\NormalTok{,}\DataTypeTok{col =} \StringTok{"black"}\NormalTok{,}
        \DataTypeTok{names.arg=}\KeywordTok{c}\NormalTok{(}\StringTok{"F1"}\NormalTok{,}\StringTok{"F2"}\NormalTok{,}\StringTok{"F3"}\NormalTok{,}\StringTok{"F4"}\NormalTok{,}\StringTok{"F5"}\NormalTok{,}\StringTok{"F6"}\NormalTok{,}\StringTok{"F7"}\NormalTok{))}
\end{Highlighting}
\end{Shaded}

\includegraphics{Statistical_Analysis_Project_files/figure-latex/unnamed-chunk-34-1.pdf}

\begin{Shaded}
\begin{Highlighting}[]
\CommentTok{#Identify columns you want, and then remove the [,1:3] below}
\KeywordTok{matplot}\NormalTok{(Maturities,Loadings[,}\DecValTok{1}\OperatorTok{:}\DecValTok{3}\NormalTok{],}\DataTypeTok{type=}\StringTok{"l"}\NormalTok{,}\DataTypeTok{lty=}\DecValTok{1}\NormalTok{,}\DataTypeTok{col=}\KeywordTok{c}\NormalTok{(}\StringTok{"black"}\NormalTok{,}\StringTok{"red"}\NormalTok{,}\StringTok{"green"}\NormalTok{),}\DataTypeTok{lwd=}\DecValTok{3}\NormalTok{)}
\end{Highlighting}
\end{Shaded}

\includegraphics{Statistical_Analysis_Project_files/figure-latex/unnamed-chunk-34-2.pdf}

\begin{Shaded}
\begin{Highlighting}[]
\KeywordTok{matplot}\NormalTok{(Factors[,}\DecValTok{1}\OperatorTok{:}\DecValTok{3}\NormalTok{],}\DataTypeTok{type=}\StringTok{"l"}\NormalTok{,}\DataTypeTok{col=}\KeywordTok{c}\NormalTok{(}\StringTok{"black"}\NormalTok{,}\StringTok{"red"}\NormalTok{,}\StringTok{"green"}\NormalTok{),}\DataTypeTok{lty=}\DecValTok{1}\NormalTok{,}\DataTypeTok{lwd=}\DecValTok{3}\NormalTok{)}
\end{Highlighting}
\end{Shaded}

\includegraphics{Statistical_Analysis_Project_files/figure-latex/unnamed-chunk-34-3.pdf}
1)In the first output from the data above, the first factor has a
relative importance of 1 compared to the rest of the factors. We can see
the Loadings of the eigenvalues by looking at the coefficients for the
different model sizes. 2)The barplot indicates again that the relative
importance of the 1st factor is one, which is incredibly high compared
to the other factors. 3)The third visualization looks at how the first
three loadings effect each of the bonds. Eigenvalues are the equivalents
to our loadings. The first loading follows a squared function, the
second loading follows a quadratic function, and the third loading
follows a cubic function. The first loading remains positive for the
entirety of the visualization. The first loading seems to peak at the
USGG2YR, and then has a negative slope for the remainder of the dataset.
The second loading initially has a very strong positive slope, and it
tapers out as it approaches the USGG30YR. The third loading initially
starts as a positive effect, but it shoots down into the negatives at
around the USGG2YR. It seemingly continues to have an even stronger
negative effect on the USGG3YR.The third loading has a negative effect
on the USGG5YR,but it is much less than the two preceding maturities.
The third loading has a positive effect on both USGG10YR and USGG30Yr.
It seems that the first and third loadings attempt to counterbalance the
2nd loading at the very beggiing of the data set. As the second loading
increases its effect in a positive direction, the third loading moves to
have a neggative effect with a very strong negative slope. They both end
as positive effects at the USGG30YR. The second loading seems to have a
possitive effect at the USGG3YR, while the 3rd Loading initially becomes
negative at around the USGG6M.\\
4)In the Fourth Visualization, Factor 1 has a massive impact on the
dataset. It has a stronger impact than any of the factors for a majority
of the dataset execpt for between the index of 2500 and 5000. Even
within that range, there are several points where Factor 1 effect
exceeds that of both factor 2 and factor 3. A question that could be
asked is why does factor 1 initially have a negative effect and then a
positive effect towards the end of our data set. This visualization
occurs because during the process of PCA, a line is placed on the axis
where there is the most variation. This line doesn't necessarily point
in the right direction which is why the magniuted of factor 1 is
correct, yet it's sign is incorrect. This is why we multiple factor 1 by
-1 below.

Change the signs of the first facor and the corresponding factor loading

\begin{Shaded}
\begin{Highlighting}[]
\NormalTok{Loadings[,}\DecValTok{1}\NormalTok{]<-}\OperatorTok{-}\NormalTok{Loadings[,}\DecValTok{1}\NormalTok{]}
\NormalTok{Factors[,}\DecValTok{1}\NormalTok{]<-}\OperatorTok{-}\NormalTok{Factors[,}\DecValTok{1}\NormalTok{]}
\KeywordTok{matplot}\NormalTok{(Factors[,}\DecValTok{1}\OperatorTok{:}\DecValTok{3}\NormalTok{],}\DataTypeTok{type=}\StringTok{"l"}\NormalTok{,}\DataTypeTok{col=}\KeywordTok{c}\NormalTok{(}\StringTok{"black"}\NormalTok{,}\StringTok{"red"}\NormalTok{,}\StringTok{"green"}\NormalTok{),}\DataTypeTok{lty=}\DecValTok{1}\NormalTok{,}\DataTypeTok{lwd=}\DecValTok{3}\NormalTok{)}
\end{Highlighting}
\end{Shaded}

\includegraphics{Statistical_Analysis_Project_files/figure-latex/unnamed-chunk-35-1.pdf}

\begin{Shaded}
\begin{Highlighting}[]
\KeywordTok{matplot}\NormalTok{(Maturities,Loadings[,}\DecValTok{1}\OperatorTok{:}\DecValTok{3}\NormalTok{],}\DataTypeTok{type=}\StringTok{"l"}\NormalTok{,}\DataTypeTok{lty=}\DecValTok{1}\NormalTok{,}\DataTypeTok{col=}\KeywordTok{c}\NormalTok{(}\StringTok{"black"}\NormalTok{,}\StringTok{"red"}\NormalTok{,}\StringTok{"green"}\NormalTok{),}\DataTypeTok{lwd=}\DecValTok{3}\NormalTok{)}
\end{Highlighting}
\end{Shaded}

\includegraphics{Statistical_Analysis_Project_files/figure-latex/unnamed-chunk-35-2.pdf}

\begin{Shaded}
\begin{Highlighting}[]
\KeywordTok{plot}\NormalTok{(Factors[,}\DecValTok{1}\NormalTok{],Factors[,}\DecValTok{2}\NormalTok{],}\DataTypeTok{type=}\StringTok{"l"}\NormalTok{,}\DataTypeTok{lwd=}\DecValTok{2}\NormalTok{)}
\end{Highlighting}
\end{Shaded}

\includegraphics{Statistical_Analysis_Project_files/figure-latex/unnamed-chunk-35-3.pdf}
Now, both of Factor 1's magnitude and sign correlate with how the
predictors act within the data set.

Q:Draw at least three conclusions from the plot of the first two factors
above A: 1) There seems to be a cyclical relationship between the two
factors at the 0 mark on the x-axis and between 0 and 2 marks on the
y-axis. 2) There is also a strong negative correlation between the -10
mark and 0 on the x-axis which could be representative that when factor
1 begins to move to have a strong positive weight from an originally
negative weight, factor 2 begins to have a negative weigh on the data.
3) At 20 on the x axis, it seems that when Factor 1 is having a very
strong weight on the data, Factor 2's weight can range between -3 and
positive 1 which is quite large. 4) There also seems to be a genera
trend from the 0 to the 16 mark on the x-axis where factor 1 and factor
2 both shar a negative slope. Factor 2 only increases around 4 units
while factor 1 increases by 16. This shows that factor 1 has a much
stronger effect on the data than factor 2 for this period.

\begin{Shaded}
\begin{Highlighting}[]
\NormalTok{OldCurve<-AssignmentData[}\DecValTok{135}\NormalTok{,]}
\NormalTok{NewCurve<-AssignmentData[}\DecValTok{136}\NormalTok{,]}
\NormalTok{CurveChange<-NewCurve}\OperatorTok{-}\NormalTok{OldCurve}
\NormalTok{FactorsChange<-Factors[}\DecValTok{136}\NormalTok{,]}\OperatorTok{-}\NormalTok{Factors[}\DecValTok{135}\NormalTok{,]}
\NormalTok{ModelCurveAdjustment.1Factor<-OldCurve}\OperatorTok{+}\KeywordTok{t}\NormalTok{(Loadings[,}\DecValTok{1}\NormalTok{])}\OperatorTok{*}\NormalTok{FactorsChange[}\DecValTok{1}\NormalTok{]}
\NormalTok{ModelCurveAdjustment.2Factors<-OldCurve}\OperatorTok{+}\KeywordTok{t}\NormalTok{(Loadings[,}\DecValTok{1}\NormalTok{])}\OperatorTok{*}\NormalTok{FactorsChange[}\DecValTok{1}\NormalTok{]}\OperatorTok{+}\KeywordTok{t}\NormalTok{(Loadings[,}\DecValTok{2}\NormalTok{])}\OperatorTok{*}\NormalTok{FactorsChange[}\DecValTok{2}\NormalTok{]}
\NormalTok{ModelCurveAdjustment.3Factors<-OldCurve}\OperatorTok{+}\KeywordTok{t}\NormalTok{(Loadings[,}\DecValTok{1}\NormalTok{])}\OperatorTok{*}\NormalTok{FactorsChange[}\DecValTok{1}\NormalTok{]}\OperatorTok{+}\KeywordTok{t}\NormalTok{(Loadings[,}\DecValTok{2}\NormalTok{])}\OperatorTok{*}\NormalTok{FactorsChange[}\DecValTok{2}\NormalTok{]}\OperatorTok{+}
\StringTok{  }\KeywordTok{t}\NormalTok{(Loadings[,}\DecValTok{3}\NormalTok{])}\OperatorTok{*}\NormalTok{FactorsChange[}\DecValTok{3}\NormalTok{]}
\KeywordTok{matplot}\NormalTok{(Maturities,}
        \KeywordTok{t}\NormalTok{(}\KeywordTok{rbind}\NormalTok{(OldCurve,NewCurve,ModelCurveAdjustment.1Factor,ModelCurveAdjustment.2Factors,}
\NormalTok{                ModelCurveAdjustment.3Factors)),}
        \DataTypeTok{type=}\StringTok{"l"}\NormalTok{,}\DataTypeTok{lty=}\KeywordTok{c}\NormalTok{(}\DecValTok{1}\NormalTok{,}\DecValTok{1}\NormalTok{,}\DecValTok{2}\NormalTok{,}\DecValTok{2}\NormalTok{,}\DecValTok{2}\NormalTok{),}\DataTypeTok{col=}\KeywordTok{c}\NormalTok{(}\StringTok{"black"}\NormalTok{,}\StringTok{"red"}\NormalTok{,}\StringTok{"green"}\NormalTok{,}\StringTok{"blue"}\NormalTok{,}\StringTok{"magenta"}\NormalTok{),}\DataTypeTok{lwd=}\DecValTok{3}\NormalTok{,}\DataTypeTok{ylab=}\StringTok{"Curve Adjustment"}\NormalTok{)}
\KeywordTok{legend}\NormalTok{(}\DataTypeTok{x=}\StringTok{"topright"}\NormalTok{,}\KeywordTok{c}\NormalTok{(}\StringTok{"Old Curve"}\NormalTok{,}\StringTok{"New Curve"}\NormalTok{,}\StringTok{"1-Factor Adj."}\NormalTok{,}\StringTok{"2-Factor Adj."}\NormalTok{,}
                      \StringTok{"3-Factor Adj."}\NormalTok{),}\DataTypeTok{lty=}\KeywordTok{c}\NormalTok{(}\DecValTok{1}\NormalTok{,}\DecValTok{1}\NormalTok{,}\DecValTok{2}\NormalTok{,}\DecValTok{2}\NormalTok{,}\DecValTok{2}\NormalTok{),}\DataTypeTok{lwd=}\DecValTok{3}\NormalTok{,}\DataTypeTok{col=}\KeywordTok{c}\NormalTok{(}\StringTok{"black"}\NormalTok{,}\StringTok{"red"}\NormalTok{,}\StringTok{"green"}\NormalTok{,}\StringTok{"blue"}\NormalTok{,}\StringTok{"magenta"}\NormalTok{))}
\end{Highlighting}
\end{Shaded}

\includegraphics{Statistical_Analysis_Project_files/figure-latex/unnamed-chunk-36-1.pdf}

\begin{Shaded}
\begin{Highlighting}[]
\KeywordTok{rbind}\NormalTok{(CurveChange,ModelCurveAdjustment.3Factors}\OperatorTok{-}\NormalTok{OldCurve)}
\end{Highlighting}
\end{Shaded}

\begin{verbatim}
##               USGG3M   USGG6M   USGG2YR   USGG3YR   USGG5YR  USGG10YR  USGG30YR
## CurveChange 1.070000 1.070000 0.8900000 0.8300000 0.7200000 0.5000000 0.4700000
##             1.090063 1.041267 0.9046108 0.8248257 0.6979317 0.5531734 0.4357793
\end{verbatim}

Q: Explain how shapes of the loadings affect the adjustments using only
factor 1, factors 1 and 2, and all 3 factors. A: The graph above
compares rows 136 and 135, and aims to see how the factors affect each
of the maturities. The first factor: The first factor takes the old
curve and inreases the new curve's Y greatly. It can be seen that the
first factor under weights the new curve from 0 to 2.5 on the x axis,
and overshoots the new curve from around 5 to 30 on the x-axis. The
first and second factor and all 3 factors combined do a nearly perfect
job of fitting the new curve.

See the goodness of fit for the example of 10Y yield.

\begin{Shaded}
\begin{Highlighting}[]
\KeywordTok{cbind}\NormalTok{(Maturities,Loadings)}
\end{Highlighting}
\end{Shaded}

\begin{verbatim}
##      Maturities                                                                                 
## [1,]       0.25 0.3839609 -0.50744508  0.5298222  0.40373501  0.3860878 -0.03976285  0.026742547
## [2,]       0.50 0.3901870 -0.43946144  0.1114737 -0.40526448 -0.6787624  0.09475452 -0.090913541
## [3,]       2.00 0.4151851 -0.11112721 -0.4187873 -0.40896949  0.3787209 -0.29848638  0.490009873
## [4,]       3.00 0.4063541  0.01696988 -0.4476561  0.06433748  0.2362448  0.19760026 -0.731570606
## [5,]       5.00 0.3860610  0.23140317 -0.2462364  0.53357656 -0.2868460  0.42125768  0.438559615
## [6,]      10.00 0.3477544  0.43245979  0.1500903  0.19856539 -0.2562426 -0.73561857 -0.152627535
## [7,]      30.00 0.3047124  0.54421228  0.4979195 -0.42098839  0.2074508  0.37776687  0.009199827
\end{verbatim}

\begin{Shaded}
\begin{Highlighting}[]
\NormalTok{Model.10Y<-AssignmentData.means[,}\DecValTok{6}\NormalTok{]}\OperatorTok{+}\NormalTok{Loadings[}\DecValTok{6}\NormalTok{,}\DecValTok{1}\NormalTok{]}\OperatorTok{*}\NormalTok{Factors[,}\DecValTok{1}\NormalTok{]}\OperatorTok{+}\NormalTok{Loadings[}\DecValTok{6}\NormalTok{,}\DecValTok{2}\NormalTok{]}\OperatorTok{*}\NormalTok{Factors[,}\DecValTok{2}\NormalTok{]}\OperatorTok{+}\NormalTok{Loadings[}\DecValTok{6}\NormalTok{,}\DecValTok{3}\NormalTok{]}\OperatorTok{*}\NormalTok{Factors[,}\DecValTok{3}\NormalTok{]}
\KeywordTok{matplot}\NormalTok{(}\KeywordTok{cbind}\NormalTok{(AssignmentData[,}\DecValTok{6}\NormalTok{],Model.10Y),}\DataTypeTok{type=}\StringTok{"l"}\NormalTok{,}\DataTypeTok{lty=}\DecValTok{1}\NormalTok{,}\DataTypeTok{lwd=}\KeywordTok{c}\NormalTok{(}\DecValTok{3}\NormalTok{,}\DecValTok{1}\NormalTok{),}\DataTypeTok{col=}\KeywordTok{c}\NormalTok{(}\StringTok{"black"}\NormalTok{,}\StringTok{"red"}\NormalTok{),}\DataTypeTok{ylab=}\StringTok{"10Y Yield"}\NormalTok{)}
\end{Highlighting}
\end{Shaded}

\includegraphics{Statistical_Analysis_Project_files/figure-latex/unnamed-chunk-37-1.pdf}
As expectd the red line fits the black line because the eigenvalues
multiplied by their vectors will reproduce numbers very close to the
original data set.

PCA not manually done

\begin{Shaded}
\begin{Highlighting}[]
\NormalTok{PCA.Yields<-}\KeywordTok{princomp}\NormalTok{(AssignmentData[,}\DecValTok{1}\OperatorTok{:}\DecValTok{7}\NormalTok{])}
\KeywordTok{cbind}\NormalTok{(PCA.Yields}\OperatorTok{$}\NormalTok{loadings[,}\DecValTok{1}\OperatorTok{:}\DecValTok{3}\NormalTok{],Maturities,Eigen.Decomposition}\OperatorTok{$}\NormalTok{vectors[,}\DecValTok{1}\OperatorTok{:}\DecValTok{3}\NormalTok{])}
\end{Highlighting}
\end{Shaded}

\begin{verbatim}
##              Comp.1      Comp.2     Comp.3 Maturities                                  
## USGG3M   -0.3839609  0.50744508  0.5298222       0.25 -0.3839609 -0.50744508  0.5298222
## USGG6M   -0.3901870  0.43946144  0.1114737       0.50 -0.3901870 -0.43946144  0.1114737
## USGG2YR  -0.4151851  0.11112721 -0.4187873       2.00 -0.4151851 -0.11112721 -0.4187873
## USGG3YR  -0.4063541 -0.01696988 -0.4476561       3.00 -0.4063541  0.01696988 -0.4476561
## USGG5YR  -0.3860610 -0.23140317 -0.2462364       5.00 -0.3860610  0.23140317 -0.2462364
## USGG10YR -0.3477544 -0.43245979  0.1500903      10.00 -0.3477544  0.43245979  0.1500903
## USGG30YR -0.3047124 -0.54421228  0.4979195      30.00 -0.3047124  0.54421228  0.4979195
\end{verbatim}

\begin{Shaded}
\begin{Highlighting}[]
\KeywordTok{matplot}\NormalTok{(Maturities,PCA.Yields}\OperatorTok{$}\NormalTok{loadings[,}\DecValTok{1}\OperatorTok{:}\DecValTok{3}\NormalTok{],}\DataTypeTok{type=}\StringTok{"l"}\NormalTok{,}\DataTypeTok{col=}\KeywordTok{c}\NormalTok{(}\StringTok{"black"}\NormalTok{,}\StringTok{"red"}\NormalTok{,}\StringTok{"green"}\NormalTok{),}\DataTypeTok{lty=}\DecValTok{1}\NormalTok{,}\DataTypeTok{lwd=}\DecValTok{3}\NormalTok{)}
\end{Highlighting}
\end{Shaded}

\includegraphics{Statistical_Analysis_Project_files/figure-latex/unnamed-chunk-38-1.pdf}

\begin{Shaded}
\begin{Highlighting}[]
\KeywordTok{matplot}\NormalTok{(PCA.Yields}\OperatorTok{$}\NormalTok{scores[,}\DecValTok{1}\OperatorTok{:}\DecValTok{3}\NormalTok{],}\DataTypeTok{type=}\StringTok{"l"}\NormalTok{,}\DataTypeTok{col=}\KeywordTok{c}\NormalTok{(}\StringTok{"black"}\NormalTok{,}\StringTok{"red"}\NormalTok{,}\StringTok{"green"}\NormalTok{),}\DataTypeTok{lwd=}\DecValTok{3}\NormalTok{,}\DataTypeTok{lty=}\DecValTok{1}\NormalTok{)}
\end{Highlighting}
\end{Shaded}

\includegraphics{Statistical_Analysis_Project_files/figure-latex/unnamed-chunk-38-2.pdf}
As expected, these are the same graphs that were seen earlier.

Change the signs fo the first factor and factor loading again.

\begin{Shaded}
\begin{Highlighting}[]
\NormalTok{PCA.Yields}\OperatorTok{$}\NormalTok{loadings[,}\DecValTok{1}\NormalTok{]<-}\OperatorTok{-}\NormalTok{PCA.Yields}\OperatorTok{$}\NormalTok{loadings[,}\DecValTok{1}\NormalTok{]}
\NormalTok{PCA.Yields}\OperatorTok{$}\NormalTok{scores[,}\DecValTok{1}\NormalTok{]<-}\OperatorTok{-}\NormalTok{PCA.Yields}\OperatorTok{$}\NormalTok{scores[,}\DecValTok{1}\NormalTok{]}
\KeywordTok{matplot}\NormalTok{(Maturities,PCA.Yields}\OperatorTok{$}\NormalTok{loadings[,}\DecValTok{1}\OperatorTok{:}\DecValTok{3}\NormalTok{],}\DataTypeTok{type=}\StringTok{"l"}\NormalTok{,}\DataTypeTok{col=}\KeywordTok{c}\NormalTok{(}\StringTok{"black"}\NormalTok{,}\StringTok{"red"}\NormalTok{,}\StringTok{"green"}\NormalTok{),}\DataTypeTok{lty=}\DecValTok{1}\NormalTok{,}\DataTypeTok{lwd=}\DecValTok{3}\NormalTok{)}
\end{Highlighting}
\end{Shaded}

\includegraphics{Statistical_Analysis_Project_files/figure-latex/unnamed-chunk-39-1.pdf}

\begin{Shaded}
\begin{Highlighting}[]
\KeywordTok{matplot}\NormalTok{(PCA.Yields}\OperatorTok{$}\NormalTok{scores[,}\DecValTok{1}\OperatorTok{:}\DecValTok{3}\NormalTok{],}\DataTypeTok{type=}\StringTok{"l"}\NormalTok{,}\DataTypeTok{col=}\KeywordTok{c}\NormalTok{(}\StringTok{"black"}\NormalTok{,}\StringTok{"red"}\NormalTok{,}\StringTok{"green"}\NormalTok{),}\DataTypeTok{lwd=}\DecValTok{3}\NormalTok{,}\DataTypeTok{lty=}\DecValTok{1}\NormalTok{)}
\end{Highlighting}
\end{Shaded}

\includegraphics{Statistical_Analysis_Project_files/figure-latex/unnamed-chunk-39-2.pdf}

\begin{Shaded}
\begin{Highlighting}[]
\KeywordTok{matplot}\NormalTok{(}\KeywordTok{cbind}\NormalTok{(PCA.Yields}\OperatorTok{$}\NormalTok{scores[,}\DecValTok{1}\NormalTok{],AssignmentData.Output,Factors[,}\DecValTok{1}\NormalTok{]),}\DataTypeTok{type=}\StringTok{"l"}\NormalTok{,}\DataTypeTok{col=}\KeywordTok{c}\NormalTok{(}\StringTok{"black"}\NormalTok{,}\StringTok{"red"}\NormalTok{,}\StringTok{"green"}\NormalTok{),}\DataTypeTok{lwd=}\KeywordTok{c}\NormalTok{(}\DecValTok{3}\NormalTok{,}\DecValTok{2}\NormalTok{,}\DecValTok{1}\NormalTok{),}\DataTypeTok{lty=}\KeywordTok{c}\NormalTok{(}\DecValTok{1}\NormalTok{,}\DecValTok{2}\NormalTok{,}\DecValTok{3}\NormalTok{),}\DataTypeTok{ylab=}\StringTok{"Factor 1"}\NormalTok{)}
\end{Highlighting}
\end{Shaded}

\includegraphics{Statistical_Analysis_Project_files/figure-latex/unnamed-chunk-40-1.pdf}
OMG!!OUTPUT1 is the first eigenvector!!

Compare the regression coefficients from Step 2 and Step 3 with factor
loadings. First, look at the slopes for
AssignmentData.Input\textasciitilde{}AssignmentData.Output

\begin{Shaded}
\begin{Highlighting}[]
\KeywordTok{t}\NormalTok{(}\KeywordTok{apply}\NormalTok{(AssignmentData, }\DecValTok{2}\NormalTok{, }\ControlFlowTok{function}\NormalTok{(AssignmentData.col) }\KeywordTok{lm}\NormalTok{(AssignmentData.col}\OperatorTok{~}\NormalTok{AssignmentData.Output)}\OperatorTok{$}\NormalTok{coef))}
\end{Highlighting}
\end{Shaded}

\begin{verbatim}
##          (Intercept) AssignmentData.Output
## USGG3M      4.675134             0.3839609
## USGG6M      4.844370             0.3901870
## USGG2YR     5.438888             0.4151851
## USGG3YR     5.644458             0.4063541
## USGG5YR     6.009421             0.3860610
## USGG10YR    6.481316             0.3477544
## USGG30YR    6.869355             0.3047124
\end{verbatim}

\begin{Shaded}
\begin{Highlighting}[]
\KeywordTok{cbind}\NormalTok{(PCA.Yields}\OperatorTok{$}\NormalTok{center,PCA.Yields}\OperatorTok{$}\NormalTok{loadings[,}\DecValTok{1}\NormalTok{])}
\end{Highlighting}
\end{Shaded}

\begin{verbatim}
##              [,1]      [,2]
## USGG3M   4.675134 0.3839609
## USGG6M   4.844370 0.3901870
## USGG2YR  5.438888 0.4151851
## USGG3YR  5.644458 0.4063541
## USGG5YR  6.009421 0.3860610
## USGG10YR 6.481316 0.3477544
## USGG30YR 6.869355 0.3047124
\end{verbatim}

This shows that the zero loading equals the vector of intercepts of
models Y\textasciitilde{}Output1, where Y is one of the columns of
yields in the data. Also, the slopes of the same models are equal to the
first loading.

Check if the same is true in the opposite direction: is there a
correspondence between the coefficients of models
Output1\textasciitilde{}Yield and the first loading? Yes there is, as
shall be seen below.

\begin{Shaded}
\begin{Highlighting}[]
\NormalTok{AssignmentData.Centered<-}\KeywordTok{t}\NormalTok{(}\KeywordTok{apply}\NormalTok{(AssignmentData,}\DecValTok{1}\NormalTok{,}\ControlFlowTok{function}\NormalTok{(AssignmentData.row) AssignmentData.row}\OperatorTok{-}\NormalTok{PCA.Yields}\OperatorTok{$}\NormalTok{center))}
\KeywordTok{dim}\NormalTok{(AssignmentData.Centered)}
\end{Highlighting}
\end{Shaded}

\begin{verbatim}
## [1] 8300    7
\end{verbatim}

\begin{Shaded}
\begin{Highlighting}[]
\KeywordTok{t}\NormalTok{(}\KeywordTok{apply}\NormalTok{(AssignmentData.Centered, }\DecValTok{2}\NormalTok{, }\ControlFlowTok{function}\NormalTok{(AssignmentData.col) }\KeywordTok{lm}\NormalTok{(AssignmentData.Output}\OperatorTok{~}\NormalTok{AssignmentData.col)}\OperatorTok{$}\NormalTok{coef))}
\end{Highlighting}
\end{Shaded}

\begin{verbatim}
##           (Intercept) AssignmentData.col
## USGG3M   1.420077e-11           2.507561
## USGG6M   1.421187e-11           2.497235
## USGG2YR  1.419747e-11           2.400449
## USGG3YR  1.419989e-11           2.455793
## USGG5YR  1.419549e-11           2.568742
## USGG10YR 1.420297e-11           2.786991
## USGG30YR 1.420965e-11           3.069561
\end{verbatim}

To recover the loading of the first factor by doing regression, use all
inputs togehter.

\begin{Shaded}
\begin{Highlighting}[]
\KeywordTok{t}\NormalTok{(}\KeywordTok{lm}\NormalTok{(AssignmentData.Output}\OperatorTok{~}\NormalTok{AssignmentData.Centered)}\OperatorTok{$}\NormalTok{coef)[}\OperatorTok{-}\DecValTok{1}\NormalTok{]}
\end{Highlighting}
\end{Shaded}

\begin{verbatim}
## [1] 0.3839609 0.3901870 0.4151851 0.4063541 0.3860610 0.3477544 0.3047124
\end{verbatim}

\begin{Shaded}
\begin{Highlighting}[]
\NormalTok{PCA.Yields}\OperatorTok{$}\NormalTok{loadings[,}\DecValTok{1}\NormalTok{]}
\end{Highlighting}
\end{Shaded}

\begin{verbatim}
##    USGG3M    USGG6M   USGG2YR   USGG3YR   USGG5YR  USGG10YR  USGG30YR 
## 0.3839609 0.3901870 0.4151851 0.4063541 0.3860610 0.3477544 0.3047124
\end{verbatim}

This means that the factor is a portfolio of all input variables with
weights!


\end{document}
